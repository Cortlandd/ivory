\section{Conclusions and Future Work}
\label{sec:conclusion}

In this paper, we have described a full-featured EDSL for high-assurance
embedded systems programming. Ivory's type system ensures safe C development,
and being an EDSL, it allows programmers the flexibility to create high-level
constructs in a type-safe fashion. This feature is used extensively in the
development of the SMACCMPilot project, allowing common idioms to be abstracted.
We have demonstrated the feasibility of developing large embedded systems in
Ivory ourselves, and there is a growing user community.

As Ivory's type system is embedded in GHC's, the properties that Ivory's type
system can encode is limited to what can be expressed in GHC: for instance,
while procedures are guaranteed to be consistent in their return type, that they
use the return statement must be checked during a separate phase.  In practice,
this limitation results in the discovery of errors later in the compilation
pipeline than would be the case in a standalone compiler.  Conversely, Ivory can
take advantage of new developments in GHC's type system.  For instance, there
are plans to integrate SMT solving into GHC's constraint solver, which would
enable more expressive array operations in the Ivory core language, as well as
enabling a richer set of derived operations.

Use of Ivory has exposed a number of avenues for future work. As mentioned in
\autoref{sec:semantics}, we are investigating the addition of nested references.
We also plan to investigate decoupling regions from function bodies, thus giving
finer-grained control over memory lifetimes.  Finally, we are considering making
regions first-class, allowing allocation to take place in a parent region.  On
the verification side, we are considering developing a weakest-precondition
style verification tool for Ivory programs, and extending the assertion language
with separation-logic predicates.
%Furthermore, Ivory's restricted core language should lend itself to static
%analysis, 
