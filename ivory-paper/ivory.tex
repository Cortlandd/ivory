%-----------------------------------------------------------------------------
%
%               Template for sigplanconf LaTeX Class
%
% Name:         sigplanconf-template.tex
%
% Purpose:      A template for sigplanconf.cls, which is a LaTeX 2e class
%               file for SIGPLAN conference proceedings.
%
% Guide:        Refer to "Author's Guide to the ACM SIGPLAN Class,"
%               sigplanconf-guide.pdf
%
% Author:       Paul C. Anagnostopoulos
%               Windfall Software
%               978 371-2316
%               paul@windfall.com
%
% Created:      15 February 2005
%
%-----------------------------------------------------------------------------


\documentclass{sigplanconf}

% The following \documentclass options may be useful:

% preprint      Remove this option only once the paper is in final form.
% 10pt          To set in 10-point type instead of 9-point.
% 11pt          To set in 11-point type instead of 9-point.
% authoryear    To obtain author/year citation style instead of numeric.

\usepackage{ifthen}
\usepackage{color}
\usepackage{framed}
\usepackage{paralist}
\usepackage{mathtools}
\usepackage{listings}
\usepackage{textcomp}
\usepackage{fixltx2e}
\usepackage{url}
\usepackage{flushend}
\usepackage{bold-extra}
%% \usepackage{array} %sjw, for <{$} etc.
\usepackage{hyperref} %sjw, for \autoref
\usepackage{stmaryrd} %sjw, for oxford brackets (\llbracket, \rrbracket)
\usepackage{amssymb}  %sjw, mainly for mathbb

%% For inference rules
\usepackage{mathpartir}

% Language definition for both quasi-quoted Ivory, and Haskell.
\lstdefinelanguage{Ivory}
{morekeywords={struct,data,where,let,in,class,instance,type,family},
 sensitive=true,
 morestring=[b]",
 escapeinside={\%*}{*)},
 commentstyle=\ttfamily,
 morecomment=[l]{--},
 morecomment=[s]{\{-}{-\}}
}

\lstset{language=Ivory}

\newboolean{td}
  \setboolean{td}{true} % modify here
  \ifthenelse{\boolean{td}}
             {\newcommand{\pat}[1]{\textcolor{blue}{PH: #1}}
              \newcommand{\lee}[1]{\textcolor{blue}{LP: #1}}
              \newcommand{\sjw}[1]{\textcolor{blue}{SW: #1}}
              \newcommand{\trevor}[1]{\textcolor{blue}{TE: #1}}
              \newcommand{\james}[1]{\textcolor{blue}{JB: #1}}
             }
             {\newcommand{\pat}[1]{}
              \newcommand{\lee}[1]{}
              \newcommand{\sjw}[1]{}
              \newcommand{\trevor}[1]{}
              \newcommand{\james}[1]{}
             }

\newcommand{\mytilde}{\raise.17ex\hbox{$\scriptstyle\mathtt{\sim}$}}
\lstnewenvironment{code}[1][]
  {\lstset{basicstyle=\scriptsize\ttfamily,#1}}
  {}
%% \lstnewenvironment{smcode}[1][]
%%   {\lstset{basicstyle=\scriptsize\ttfamily,#1}}
%%   {}

%% \newenvironment{cols}{\begin{tabular}{m{3.6cm}|m{3.6cm}} &
%%     \\\hline}{\end{tabular}}

\newcommand{\cd}[1]{\texttt{#1}}

\begin{document}

\special{papersize=8.5in,11in}
\setlength{\pdfpageheight}{\paperheight}
\setlength{\pdfpagewidth}{\paperwidth}

\conferenceinfo{ICFP~'14}{September 1--6, 2014, Gothenburg, Sweden}
\copyrightyear{2014}
\copyrightdata{978-1-4503-2873-9/14/09}
\doi{2628136.2628146}

% Uncomment one of the following two, if you are not going for the 
% traditional copyright transfer agreement.

%\exclusivelicense                % ACM gets exclusive license to publish, 
                                  % you retain copyright

\permissiontopublish             % ACM gets nonexclusive license to publish
                                  % (paid open-access papers, 
                                  % short abstracts)

\titlebanner{Under submission}        % These are ignored unless
\preprintfooter{short description of paper}   % 'preprint' option specified.

\title{Ivory}
\subtitle{catchy subtitle}

\authorinfo{Trevor decides authors and order}
           {Galois, Inc.}
           {\{foobar\}@galois.com}
%% \authorinfo{Name2\and Name3}
%%            {Affiliation2/3}
%%            {Email2/3}

\maketitle

\begin{abstract}
Ivory is a ``safe C'' language, enforcing memory safety and removing most
undefined behaviors, while providing low-level control of memory-manipulation
and timing. Ivory is embedded in a modern variant of Haskell, as implemented by
the GHC compiler. The main contributions of the paper are two-fold. First, we
demonstrate how to embedded type type-system of a safe-c language into the
dependent-type extensions of GHC. Second, Ivory is of interest in its own right,
as a powerful language for writing high-assurance embedded programs. Beyond
invariants enforced by its type-system, Ivory has direct support for
model-checking, theorem-proving, and property-based testing. Ivory's semantics
have been formalized and proved correct.
\end{abstract}

\category{D.3.2}{Language Classifications}{Applicative (functional) languages}

% general terms are not compulsory anymore,
% you may leave them out
%% \terms
%% term1, term2

\keywords
Embedded Domain Specific Languages; Embedded Systems

\section{Introduction}
\label{sec:introduction}

Recent reports of car-hacking via software flaws~\cite{}, unsafe software
safety-critical software~\cite{}, and bugs in internet-of-things
devices~\cite{}, all point toward the need for safe low-level programming
languages. Languages like C or C++ are still the golden standard in embedded
system development given the low-level control they provide in terms of memory
usage and timing behavior. But C/C++ is wrought with undefined behavior with
poor support for formal analysis.

In this paper we present the language \emph{Ivory}.  Ivory follows in the
footsteps of other ``safe C'' programming languages, like
Cyclone~\cite{cyclone}, BitC~\cite{bitc}, and Rust~\cite{rust}. By safe C
languages, we mean languages that avoid many of the pitfalls of C, particularly
related to memory safety and undefined behavior, while being suitable for
writing low-level code (e.g., device drivers). Ivory, like other safe C
languages, has a minimal runtime system.

Ivory's implementation, however, is unique compared to previous safe C
languages. Ivory is implemented as an embedded domain-specific langauge
(EDSL)~\cite{edsl} within Haskell. Ivory's type system is (shallowly) embedded
within Haskell's type system, including dependently-typed extensions provided by
GHC~~\cite{dephaskell}. Without writing a stand-alone type-checker, Ivory types
guarantee memory safety of Ivory programs at \emph{Haskell} compile time.

However, the Ivory language itself is deeply embedded within Haskell, easing the
ability to develop various back-ends and verification tools. Not only does Ivory
generate embedded C, it contains an integrated SMT-based symbolic simulator, and
a theorem-prover back-end. The combination of a shallowly-embedded types and
deeply embedded values is novel and pragmatic.

Ivory is not a toy language: we have used Ivory to write
\emph{SMACCMPilot}~\cite{smaccm}, a full-featured high-assurance
autopilot for a small unpiloted air vehicle.  Furthermore, Boeing has
used Ivory to implement a level-of-interoperability for
Stanag~4586~\cite{stanag}, a unpiloted air vehicle communications
standard. We know of a few additional small projects by other
developers in Ivory, as well.  There are well over 100~KLoC of Ivory in existence.

%% maybe remove?  I wanted to talk briefly about the EDSL aspect, etc.
%% Question this answers: why should I care?  Might be answered in the
%% next bit ...
\sjw{more of a summary paragraph, probably rephrase or remove?}

\paragraph{Contributions}
In 2014, Stephanie Weirich gave an ICFP keynote describing dependent programming
using recent type and language extensions of GHC's implementation of Haskell
(which we will colloquially refer to as ``GHC'' in the remainder of this
paper)~\cite{weirich-keynote}. Weirich describes how recent extensions to GHC
provide much of the power of dependently-typed programming, such as found in
Agda~\cite{agda}, Idris~\cite{idris}, or Coq~\cite{coq}. However, in GHC, a
surprisingly powerful subset of dependent typing features can be used while
still enjoying type-inference and decidable type-checking~\cite{dephaskell}.

Ivory exemplifies the use of GHC dependent types in a large, fully-featured
EDSL. The language demonstrates that type checking for safe systems programming
can be embedded into GHC's type system, verifying properties involving
pointers, arrays, loops and local memory allocation. Indeed, Ivory's type
system goes beyond mere safety and tracks control-flow effects at host-language
compile-time.

After providing a brief introduction to the Ivory language in
Section~\ref{sec:ivory-overview}, we describe Ivory's embedding in GHC's type
system in Section~\ref{sec:ivory-embedding}. We highlight the aspects of the
language particularly relevant to memory-safety (e.g., pointers, structures, and
memory allocation). We also highlight shortcomings of the approach, describing
aspects of the language that cannot be checked by the host language's type
system (e.g., Ivory's module system).

Embedding a type system for a safe C language into GHC's type system is tricky
business. To gain confidence that our embedding is correct, we have formalized a
model of Ivory in the Isabelle theorem prover~\cite{isabelle}, and used the model to
formally prove progress and preservation properties for Ivory. In the process,
we have discovered minor bugs in Ivory's type embedding in GHC as well as
generalizations to Ivory that still preserve safety. We describe the
formalization, proofs, and extensions in Section~\ref{sec:semantics}.

Ivory goes beyond ensuring memory safety, the focus on most other safe C
programming languages, and also provides automated support for preventing errors
that result from other undefined behaviors in C (e.g., division by zero, left
bit-shifts by a negative value, etc.) as well as support for checking
user-provided assertions. Toward this end, Ivory supports writing user-supplied
assertions and pre- and post-conditions on functions, and includes a built-in
symbolic simulator targeting an SMT solver (CVC4~\cite{cvc4}), as well as an
theorem-prover back-end targeting ACL2~\cite{acl2}. For automated testing, a
QuickCheck-like property-based test-case generator is integrated into
Ivory. These tools are described in Section~\ref{sec:tools}.

We describe related work, in safe C language and EDSLs in
Section~\ref{sec:related-work} and provide concluding remarks in
Section~\ref{sec:conclusion}.

Ivory is open-source (BSD3 license) and available at \url{ivory-lang.org}.


\section{Ivory Overview}
\label{sec:ivory-overview}

\lee{need lead-in here. What does Ivory do? What's the motivation for it's
  design (I like to motivate it with the JPL's Power of 10 rules).}

In this section, we give an illustrative overview of Ivory.  An Ivory
program is a Haskell program producing a collection of Ivory modules,
each module containing type and procedure definitions.  %% Type
%% definitions are produced using a quasi-quoter\cite{quoted}, while procedure
%% definitions are built from a set of monadic combinators.

Ivory is a staged language: the Haskell program compiles Ivory modules
to produce an AST which is then passed to one or more back-ends.  Thus,
an executable is produced from an Ivory program by compiling and
running the Haskell code to produce a C program, which is then
compiled with a C compiler.  Alternately, checking of pre- and
post-conditions is performed by running the Haskell program in
conjunction with the verification back-end.

In the following, we introduce both the types and values of Ivory programs but
postpone most discussion of the types to Section~\ref{sec:ivory-embedding}. We
focus on core aspects of the language in this introduction and throughout
the paper. Ivory contains a large number of operators and standard libraries we
elide. Examples include serialization, safe type casts, nullable pointers (for
inter-operation with legacy C), function pointers, and bit operators.

\subsection{Ivory Statements}

Ivory statements are terms in the \cd{Ivory} monad.  This monad
provides fresh variables, along with constructors for Ivory
statements. Unlike C, Ivory expressions must be pure, so
side-effecting operations take place at the statement level, in
the context of the monad. This ensures a defined order for effects,
eliminating large classes of unintuitive and undefined behaviors.

Memory in Ivory is manipulated via non-nullable references~\cite{memareas}.
References are read and written using the \cd{deref} and \cd{store}
statements, respectively.  For example, the following Haskell function
takes a reference to a signed 32-bit integer value and constructs
a program fragment which increments the value of the reference.

\begin{code}
incr_ref :: Ref s (Stored Sint32) -> Ivory eff ()
incr_ref r = do
    v <- deref r
    store r (v + 1)
\end{code}

\noindent
A reference in Ivory may refer either to a global object, allocated at
compile time, or a \emph{local} object, allocated dynamically.
Dynamic objects are created in ephemeral regions associated
with the scope of the containing procedure; operationally, local
objects are allocated on the stack, so regions are implicitly freed
on procedure return.  Ivory reference types are indexed by region
variables, the parameter \cd{s} seen in the type signatures above.
Along with type variable scoping, these region annotations on
references ensure that references do not escape the context in which
they were allocated.

The definition \cd{incr\_ref} is not a complete Ivory procedure.
Rather, it can be thought of as a template parameterised by a reference.
Ivory procedures must be explicitly defined and exported through
procedure definitions, such as
\begin{code}
incr_def :: Def ('[Ref s (Stored Sint32)] :-> Sint32)
incr_def = proc "incr_def" $ \r -> body $ do
  incr_ref r
  v <- deref r
  ret v
\end{code}
\noindent
The procedure \cd{incr\_def}, introduced by the use of \cd{proc}, calls the
Haskell function \cd{incr\_ref} above.  The use of \cd{body} indicates that the
procedure's definition will follow.  A more thorough treatment of \cd{proc} and
\cd{body} is given in section \ref{sec:proc}.  The type of \cd{incr\_def} makes
use of the data kinds extension, representing the argument list as a promoted
list.  The \cd{ret} statement returns a value from the Ivory procedure (we use
\cd{ret} rather than \cd{return} to avoid conflicting with Haskell's \cd{return}
function).

%% \nonident
%% The Ivory embedding uses standard Haskell techniques\cite{} to reuse
%% Haskell's binders to name procedure arguments.  These binders do not
%% appear in the AST, however, being erased by the \cd{proc} smart
%% constructor.

The Ivory monad tracks effects (the \cd{eff} type parameter); see
Section~\ref{sec:ivory-monad}.  One of these effects is the current allocation
region: the allocation function \cd{local} returns a reference that is tied to
that region.  For example, the following constructs a zero initialized reference
to an integer; the \cd{ival} constructs an initializer from a value:

\begin{code}
make_zero :: (GetAlloc eff %*\mytilde*) Scope s)
          => Ivory eff (Ref s (Stored Sint32))
make_zero = local (ival 0)
\end{code}

\subsection{Data structures}

Ivory provides C-style arrays and structures.  Array types are
parameterised by their size and the type system ensures that array accesses are
within bounds. Structures are defined using a quasi-quoter to specify the
field names and their types. Arrays and structures belong to a special kind,
\cd{Area}, whose values are only ever manipulated through references.  In order
to lift the types of values that can be manipulated directly to \cd{Area}, the
\cd{Stored} type constructor is used.  Fields in structures are accessed
via the \cd{\mytilde>} operator which takes a reference to a struct and a field
name, and returns a reference at the field's type. For example, the following
code declares a structure type named \cd{position}, as well as a function that
will allocate an instance, and perform some basic operations on it.

\begin{code}
[ivory|
struct position
  { latitude  :: Stored IFloat
  ; longitude :: Stored IFloat
  ; altitude  :: Stored Sint32
  }
|]
struct_ex = do
  s <- local (istruct [ latitude  .= ival 45.52
                      , longitude .= ival (-122.68)
                      , altitude  .= ival 1524 ])
  lat <- deref (s %*\mytilde*)> latitude)
  lon <- deref (s %*\mytilde*)> longitude)
  incr_ref (s %*\mytilde*)> altitude)
\end{code}

\subsection{Control structures}
\label{sec:control}

Ivory supports the usual control structures: the \cd{ifte\_} statement
constructor takes a boolean argument and two statements, one for each
branch of the if-then-else, while the pure ternary operator, \cd{?},
selects from two alternatives at the expression level.

\begin{code}
abs :: Def('[Sint16] :-> Sint16)
abs = proc "abs" $ \v -> body $ do
  ifte_ (v <? 0)
    (ret (-1*v))
    (ret v)

abs2 :: Def('[Sint16] :-> Sint16)
abs2 = proc "abs2" $ \v -> body $ do
  ret $ (v <? 0) ? ((-1*v), v)
\end{code}
%$
Ivory has two classes of iteration constructs: \cd{forever} for non-terminating
loops such as OS tasks, and loops with constant bounds. The prototypical
bounded loop in Ivory is the \cd{arrayMap}, which iterates over the elements of
an array. For example, the following
procedure adds \cd{x} to each element of the array \cd{arr}, noting
that \cd{arr ! ix} returns a \emph{reference} to the \cd{ix}-th
element of \cd{arr}.

\begin{code}
mapProc = proc "mapProc"
        $ \arr x -> body
        $ arrayMap
        $ \ix -> do
            v <- deref (arr ! (ix :: Ix 4))
            store (arr ! ix) (v + x :: Uint8)
\end{code}
%$
Note that we do not need to pass \cd{arr} to \cd{arrayMap} to determine the
correct bounds on the loop; rather, as we explain in \autoref{sec:area}, GHC can
\emph{infer} the bounds from the loop body!

\subsection{Assertions}

Ivory supports pre- and post-conditions, along with assertions.  The
Ivory compiler can emit run-time assertions to enforce these
conditions, or the model checker back-end can be used to statically
verify these properties hold.

\begin{code}
predicates_ex :: Def('[ IFloat ] :-> IFloat)
predicates_ex = proc "predicates_ex" $
    \i -> requires (i >? 0)
        $ ensures (\r -> r >? 0)
        $ body
        $ do (assert (i /=? 0))
             ret (i + 1))
\end{code}

\noindent
An \cd{ensures} clause takes a function, such that when applied to the return
value at any return point in the procedure, the predicate should hold.

\sjw{removed, what is the main point of this section?}
\jamey{The printf\_proc example illustrates this, but does it pass the sanity
checker? I don't think so...?}
\eric{Good point, the sanity checker will complain about the call to
  printf\_none, but I'd call this a shortcoming of the sanity checker. I guess
  it needs some notion of sum types for importProcs (and maybe externProcs?).}

% \begin{code}

% print_ex_module :: Module
% print_ex_module = package "print_ex" $ do
%   incl printf_none
%   incl printf_sint32
%   incl print_proc

% printf_none :: Def('[IString] :-> Sint32)
% printf_none  = importProc "printf" "stdio.h"

% printf_sint32 :: Def('[IString, Sint32] :-> Sint32)
% printf_sint32  = importProc "printf" "stdio.h"

% print_proc = Def('[]:->())
% print_proc = proc "print_proc" $ body $ do
%   _ <- call printf_none "hello, world!\n"
%   _ <- call printf_sint32 "print an integer: \%d" 42
%   return ()
% \end{code}

% Ivory can interact with externally defined C functions and global
% variables. The \cd{importProc} primitive allows the user to declare an external
% procedure, and ensures the correct header file is included by the generated
% code.

% Ivory can only import and use functions that have a valid Ivory type signature.
% Some polymorphic C functions may have multiple valid Ivory types.
% \jamey{The printf\_proc example illustrates this, but does it pass the sanity
% checker? I don't think so...?}

% \subsection{Toolchain Use}
% The Ivory compiler is a Haskell function that takes a list of \cd{Module}s,
% parses command line options, and writes generated C source and header files to
% a directory given by those options.

% The compiler's second argument is a list of \cd{Artifact}s. Artifacts are a
% datatype for an arbitrary Haskell string and a filename, indicating the contents
% of a non-Ivory-generated file to be written to the output directory. In
% practice, this is used to write Makefiles, native C sources, linker scripts, and debug output
% specified by the user.

% There are also related functions exposed to the user that allow the parsing
% of command line options to be separated from the compile step, where desired.

% \begin{code}
% import Ivory.Compile.C (compile)

% main :: IO ()
% main = compile [ ex_module, print_ex_module ] []
% \end{code}







\section{Ivory Embedding}
\label{sec:ivory-embedding}

In this section, we describe the implementation of Ivory, focusing on embedding
the Ivory type system in the GHC type system.

\subsection{The Ivory Monad}
\label{sec:ivory-monad}

Ivory statements have the type

\begin{code}
Ivory (eff :: Effects) a
\end{code}

\noindent
This type is a wrapper for a writer monad transformer over a state monad. The writer
monad writes statements into the Ivory abstract syntax tree, and the
state monad is used to generate fresh names for variables.

\lee{note in particular that effects can be ``turned on'' and ``turned off''}
\paragraph{Effects}
The \cd{eff} type parameter is a phantom type that tracks
effects at the type level. (These effects have no relation to
the recent work on effects systems for monad transformers~\cite{effects}.)
Currently, we track three classes of effects for Ivory statement blocks:

\begin{itemize}
\item \emph{Returns}: does the code block contain a \cd{ret} statement, and
  is the type of the returned value correct?
\item \emph{Breaks}: does the code block contain a \cd{break} statement?
\item \emph{Allocates}: does the code block contain local memory allocation?
\end{itemize}

Intuitively, these effects matter because their safety depends on the context in
which the monad is used. For example, a \cd{ret} statement is safe when used
within a procedure, to implement a function return. But an Ivory code block can
also be used to implement an operating system task that should never
return. Similarly in Ivory, \cd{break} statements are used to terminate
execution of an enclosing loop. (The other valid use of \cd{break} in C99 is to
terminate execution in a \cd{switch} block, but Ivory does not contain
\cd{switch}.) By tracking break effects, we can ensure that an Ivory block
containing a \cd{break} statement is not used outside of a loop. Finally,
allocation effects are used to guarantee that a reference to locally-allocated
memory is not returned by a procedure, which would result in undefined behavior; see
Section~\ref{sec:ref} for details. Moreover, we can prohibit a code block from
allocating memory simply by removing allocation effects from its type.

The Ivory effects system is implemented by a type-level tuple
where each of the three effects correspond to a field of the
tuple. Type equality constraints enforce that a particular effect is (or is not)
allowed in a given function signature.

We use GHC's data kinds extension~\cite{data-kinds} to lift the following type
declaration to a \emph{kind} declaration.

\begin{code}
data Effects = Effects ReturnEff BreakEff AllocEff
\end{code}

\noindent
The individual effect types are implemented similarly, using GHC to derive a
kind from the type definition. For example, the \cd{BreakEff} type/kind
describes whether a break statement is allowed in a block of statements.

\begin{code}
data BreakEff = Break | NoBreak
\end{code}

\noindent
Type families~\cite{typefamilies} are used to access and modify the types at each field of
the tuple. For example, the \cd{GetBreaks} family extracts the \cd{BreakEff}
field of an \cd{Effects} tuple.
\footnote{The GHC syntax is to precede a data kind type constructor with a tick (\cd{'})
to disambiguate it from the corresponding data constructor.}

\begin{code}
type family   GetBreaks (effs :: Effects) :: BreakEff
type instance GetBreaks ('Effects r b a) = b
\end{code}

\noindent
The \cd{AllowBreak} and \cd{ClearBreak} families turn the effect ``on'' or
``off'', respectively.

\begin{code}
type family   AllowBreak (effs :: Effects) :: Effects
type instance AllowBreak ('Effects r b a) =
    'Effects r 'Break a

type family   ClearBreak (effs :: Effects) :: Effects
type instance ClearBreak ('Effects r b a) =
    'Effects r 'NoBreak a
\end{code}

With this machinery, we can now use a type equality constraint to enforce
the particular effects in a context. For example, Ivory's \cd{break} statement
has the type

\begin{code}
break :: (E.GetBreaks eff %*\mytilde*) E.Break) => Ivory eff ()
break = ...
\end{code}

\subsection{Types}
\label{sec:types}

Ivory uses two type classes to define its domain: \cd{IvoryType} and
\cd{IvoryArea}.  \cd{IvoryType} classifies all types that make up valid Ivory
programs.  As Ivory programs build up the AST of the program they represent when
they are run, this class describes the set of types that contain fragments of
the Ivory AST.  The \cd{IvoryArea} class serves to ensure that primitive types
that are stored in references also have an instance of \cd{IvoryType}.  Types
that have \cd{IvoryType} instances include signed and unsigned integers, the
void type \cd{()}, and references, while types that have an \cd{IvoryArea}
instance are limited to those that have kind \cd{Area}, defined in
section~\ref{sec:area}.  All types used in Ivory programs will have an
\cd{IvoryType} or \cd{IvoryArea} instance.

\begin{figure}[ht]
\begin{code}
class IvoryType t
class IvoryType t => IvoryVar a where
  unwrapExpr :: t -> Expr
  wrapVar    :: Var -> t
class IvoryVar t => IvoryExpr t
  wrapExpr   :: Expr -> t

class    IvoryArea (area :: Area *)
instance IvoryType t => IvoryArea (Stored t)
instance IvoryArea ...
\end{code}
\caption{Classes used to define Ivory's domain}
\label{fig:types}
\end{figure}

The \cd{IvoryVar} and \cd{IvoryExpr} class further stratify Ivory types that
have values.  The \cd{IvoryVar} class, which is a superclass of \cd{IvoryExpr},
describes all types that can have an Ivory expression extracted from them, as
well as be created from a fresh name.  This roughly corresponds to types that
can be used as an L-value in assignments as well as formal parameters.  The
\cd{IvoryExpr} class includes types that can be constructed from full
expressions, and corresponds to the set of types whose values can be used in the
position of an R-value.  It might be tempting to say that the functionality of
the \cd{IvoryVar} belongs in the \cd{IvoryType} class. However, Ivory has a void
type (\cd{()}) so we do require this distinction to prevent void values from
being created.  Most types used in Ivory provide instances for all three
classes, \cd{IvoryType}, \cd{IvoryVar}, and \cd{IvoryExpr}, with only a few
exceptions like \cd{()} defining a subset.  See Figure~\ref{fig:types} for the
relationship between these classes.


\subsection{Memory Management}
\label{sec:ref}

Ivory uses regions for memory management~\cite{memareas}.  When data is allocated, a reference
to the resulting data is returned, and tagged by the containing region using a
type variable.  Well-typed Ivory programs guarantee that references do no
persist beyond the scope of their containing region.  Regions in Ivory classify
global data, and data allocated and freed on procedure entry/exit (the back-end
relies on stack-based allocation in C), through the language could be modified
to separate the two concepts, as discussed in section~\ref{sec:sem-discuss}.
Corresponding to these two kinds of regions are the region tags that Ivory
supports: \cd{Global} which holds statically-allocated global data that is
available for the lifetime of the program, and a local region unique to each
procedure whose lifetime is tied to that of the procedure.

Data with \cd{Global} scope is allocated through the use of the \cd{area}
top-level declaration, then converted to a reference through the use of the
\cd{addrOf} function.  As the \cd{area} function produces a top-level
declaration, it also requires a symbol to use as the name of the allocated
memory.  Data allocated within a procedure is allocated through the use of the
\cd{local} function, and are tagged with the region of that function.  Since
procedure definition introduces a fresh region, with the constraint that no
reference allocated in that region should show up in the return type of that
procedure, Ivory avoids the introduction of dangling pointers.  The embedding of
this feature in Haskell will be described in more detail in
section~\ref{sec:proc}.

Both forms of allocation take initializers, though \cd{Global} allocation
through \cd{area} will default to zero-initialization if it is omitted.
Initializers are functions that embed values into a structure that mirrors that
of a memory area.  As an example, the \cd{example} value in
Figure~\ref{fig:mem-alloc} defines an initializer for an array of three
\cd{Uint8} values.  The types of the allocation functions, as well as a sample
of the initializers available are given in Figure~\ref{fig:mem-alloc}.

\begin{figure}[ht]
\begin{code}
local  :: (GetAlloc eff %*\mytilde*) 'Scope s, IvoryArea area)
       => Init area -> Ivory eff (Ref s area)

data MemArea (area :: Area *)
area   :: (IvoryArea area)
       => Sym -> Maybe (Init area) -> MemArea area
addrOf :: (IvoryArea area)
       => MemArea area -> Ref Global area

data Init (area :: Area *)
izero  :: IvoryZero area => Init area
ival   :: IvoryType val  => val -> Init (Stored val)
iarray :: IvoryArea area
       => [Init area] -> Init (Array n area)

example :: Init (Array 3 (Stored Uint8))
example  = iarray (map ival [1,2,3])
\end{code}
\caption{Memory allocation and initialization functions}
\label{fig:mem-alloc}
\end{figure}

As noted in section~\ref{sec:ivory-monad}, allocation is tracked through an
effect in the effect context of the Ivory monad.  The result of this is that
each call to the \cd{local} allocation function produces references that are
tied to that specific context.  Conversely, if the current effect context has
no allocation scope, there is no way to produce a new reference.  As allocation
that takes place at the top-level is implicitly in the \cd{Global} region, there
is no need to involve the Ivory monad.

Once a reference has been acquired, it may be stored to and read from in the
context of the Ivory monad using the \cd{store} and \cd{deref} functions.  The
Ivory monad does not track effects for manipulating specific references, and
instead allows reading and writing to any reference that is in scope, within the
context of the Ivory monad.

\subsection{Memory Areas}
\label{sec:area}

In addition to being parameterized on the region they are allocated in,
references are parameterized by the layout of the memory they point to.  We
introduce area types through the \cd{Area}\footnote{The type parameter on the
\cd{Area} kind is present so that when giving kind-signatures, we can fix the
kind of stored-values as being star (\cd{*})-kinded.  As Haskell currently lacks
a construct for defining kinds without data, this parameterization is necessary,
as kinds are specified with a syntax that is invalid where a type is expected.
This technique was described by Magalh\~{a}es~\cite{jpm:trkgp:12}.} kind, and
the four types that inhabit it (Figure~\ref{fig:area-def}).  This typing of
memory is heavily inspired by the work of Diatchki and Jones~\cite{memareas}.
Ivory supports four kinds of areas that we explain below: arrays with statically
known size, ``C'' arrays without statically known size (for communicating with
external C functions), structs, and stored atomic values.

\begin{figure}[ht]
\begin{code}
data Area k = Array Nat (Area k)
            | CArray (Area k)
            | Struct Symbol
            | Stored k


store :: IvoryStore a
      => Ref s (Stored a) -> a -> Ivory eff ()
deref :: IvoryStore a
      => Ref s (Stored a) -> Ivory eff a

data Label (struct :: Symbol) (area :: Area *)
(%*\mytilde*)>) :: Ref s (Struct sym) -> Label sym a -> Ref s a
(!)  :: Ref s (Array n area) -> Ix n -> Ref s area
\end{code}
\caption{The definition of the \cd{Area} kind, and associated operations}
\label{fig:area-def}
\end{figure}

\paragraph{Stored values}
The simplest type of memory area is a single base type, lifted to the \cd{Area}
kind by the use of the \cd{Stored} type constructor.  For example, the area type
of a \cd{Sint32} would simply be \cd{Stored Sint32}. The \cd{store} and
\cd{deref} operators will only operate over references that point to \cd{Stored} areas,
mirroring the operations from~\cite{memareas}, as this allows us to never deal
directly with a value of type \cd{Array}, or \cd{Struct}; we only ever read and
write references to values, never references to aggregate values.

As the \cd{Stored} area-type allows the lifting of any star-kinded type to a
memory area, we constrain the operations on references to restrict what is
storable.  This constraint is enforced via the \cd{IvoryStore} class.  While the
\cd{IvoryStore} constraint is used to rule out most types from being stored in a
reference, it is worth noting that it is also used to prohibit the storing of other
references.  The reason for this restriction is twofold:

\begin{enumerate}
\item We allow the use of default initializers during allocation, but do not have
  a good way to say what parts of a structure are required, thus potentially
  introducing a null reference when initializing structures that contain
  references.
\item As there is currently no connection between the region of a reference, and
  the region of any references it points to, it would be possible to persist a
  reference beyond its lifetime by storing it in a longer-lived reference.
\end{enumerate}

\paragraph{Structs} A reference that has an area-kind of type \cd{Struct "x"}
will point to memory whose layout corresponds to the definition of the struct
with name ``x''.  Struct definitions are introduced through use of the ivory
quasi-quoter~\cite{quoted}.  For example, if a region of memory is typed using
the following struct declaration, it would have type \cd{Struct "a"}.

\begin{code}
[ivory| struct a { field1 :: Stored Sint32
                 , field2 :: Struct "b"
                 }
|]
\end{code}

Also introduced by the struct declaration are field labels.  Field labels allow
for indexing into a memory area, producing a reference to the value contained
within the struct.  For example, using the previous struct definition, the
quasi-quoter introduces two labels, \cd{field1} and \cd{field2}, for accessing
those fields given a reference to an ``\cd{a}'' struct:

\begin{code}
field1 :: Label "a" (Stored Sint32)
field2 :: Label "a" (Struct "b")
\end{code}

Using a struct label to select the field of a structure requires the use of the
\cd{(\mytilde>)} operator, which expects a reference to a structure as its first
argument, and a compatible label as its second.  The type of the
\cd{(\mytilde>)} operator is given in Figure~\ref{fig:area-def}.  In the
following example, the \cd{(\mytilde>)} operator is used with a reference to an
``\cd{a}'' struct, with the \cd{field1} label, producing a new reference of type
\cd{Ref Global (Stored Sint32)}.

\begin{code}
example :: Ref Global (Struct "a")
        -> Ref Global (Stored Sint32)
example ref = ref %*\mytilde*)> field1
\end{code}

Operations for indexing are pure in Ivory, as they only manipulate a base
pointer; the value of a reference is never dereferenced until an explicit use of
the \cd{deref} primitive, which is an effectful operation.

\paragraph{Arrays}

\begin{figure}[ht]
\begin{code}
arrayMap :: (Ix n -> Ivory (E.AllowBreak eff) a)
         => (Ix n -> Ivory eff a) -> Ivory eff a
arrayLen :: Num len => Ref s (Array n area) -> len
toCArray :: Ref s (Array n area)
         -> Ref s (CArray area)
\end{code}
\caption{Array support functions.}
\label{array-support-functions}
\end{figure}

Arrays in Ivory take two type parameters: the length of the array as a
type-level natural number, and the area type of its elements.  For example, an
array of 10 signed 32-bit integers would have the type \cd{Array 10 (Stored
Sint32)}.  Indexing into arrays is accomplished through the use of the \cd{(!)}
operator, shown in Figure~\ref{array-support-functions}. Indexing an array does
not dereference it, but returns a reference to the indexed cell.

An index into an array has the type \cd{Ix}, which is parameterized by the size
of the array that it is indexing into.  The \cd{Ix n} type will only hold
values between zero and \cd{n-1}, which allows us to avoid run-time array bounds
checks~\cite{memareas}.  One shortcoming of this approach is that the
\cd{(!)} operator will only accept indexes that are parameterized by the length
of the array being indexed, while it would be useful to allow indexes that have
a maximum value that is less than the length of the target array.

As array indexes are parameterized by the length of arrays they can index into,
they become an interesting target for new combinators.  In this vein, we
introduce \cd{arrayMap}, whose signature is shown in
Figure~\ref{array-support-functions}.  The intuition for the \cd{arrayMap}
function is that it invokes the function provided for all indexes that lie
between $0$ and $n - 1$.  As the index argument given to the function is most
often used with an array, type information propagates out from uses of the
\cd{(!)} operator, and it becomes unnecessary to give explicit bounds for the
iteration. Additionally, as the size of the index is tied to the size of the
array being indexed, it is unnecessary to provide an array as an argument to
\cd{arrayMap}: we rely on the use of the index to set the bounds of the
loop. The implementation relies on type-level natural numbers being singleton
types, with the ability to construct a value $n$ inhabiting the type $n$.

For compatibility with C, we also introduce a type for arrays that are not
parameterized by their length, \cd{CArray}.  There are no operations to work
with references to \cd{CArray}s in Ivory, as the assumption is that they will
only ever be used when interacting with external C functions.  As many C
functions that consume arrays require both a pointer and a length, we also
provide the \cd{arrayLen} function, which allows the length of an Ivory array to
be demoted to a value.  When used in conjunction with \cd{toCArray}, this
function allows for fairly seamless integration with external C code.


\subsection{Procedures}
\label{sec:proc}

Ivory procedures differ from Haskell functions in that they behave as compiled
procedures, not macros; Haskell functions that produce \cd{Ivory} values will be
expanded at compile time, while Ivory procedures will be translated into
procedures in the target language.  Procedures in Ivory inhabit the \cd{Def}
type which is parameterized by the signature of the function it names.
Procedure signatures inhabit the \cd{Proc} kind, which provides one type
constructor: \cd{:->}.  The \cd{:->} type constructor takes two arguments: the
types of the argument list, and the return type of the whole procedure.  The
intent behind the use of the \cd{:->} type is to suggest that all of the
arguments to the left of the arrow must be provided before a result may be
produced.

\paragraph{Definition}

Procedures are defined through the use of the \cd{proc} function, which requires
two arguments: a symbolic name for the generated procedure and its
implementation.  The implementation takes the form of a Haskell function that
accepts Ivory value arguments, and produces a result in the Ivory monad.  Again,
viewing Haskell functions that produce values in the Ivory monad as macros, the
\cd{proc} function can be seen as operating at the meta-level, accepting a
symbol name and a macro as its arguments, and producing a procedure with the given
name, and the fully-applied macro as its body.  Correct procedure definition is
guarded by the \cd{IvoryProcDef} class, shown in Figure~\ref{fig:proc-defs},
which constrains uses of the \cd{proc} function.

\cd{IvoryProcDef} has two parameters: signature and implementation, which relate
the \cd{Proc} type of the resulting Ivory procedure and the Haskell function
given as the its implementation.  There are only two instances for
\cd{IvoryProcDef}: the case where the argument list is empty, and the case where
the argument list is extended by one argument, corresponding to the cases for
the \cd{'[]} and \cd{(':)} type constructors.  The latter case also requires
that the argument added be an Ivory type that is inhabited by a use of the
\cd{IvoryVar} constraint.  In addition to ensuring that the argument type is
acceptable as an argument to an Ivory function, the use of the \cd{IvoryVar}
constraint also allows values to be manufactured from fresh names, allowing
dummy values to be passed to the implementation function.

Examining the functional dependencies for the \cd{IvoryProcDef} class from
Figure~\ref{fig:proc-defs}, we see that the implementation function (\cd{impl})
determines the signature of the resulting procedure (\cd{sig}).  The effect of
this dependency in the context of the \cd{proc} function is that the user will
rarely need to write an accompanying \cd{Def} signature for Ivory procedures
they define; uses of the arguments to a procedure will often yield a monomorphic
implementation function, which through the functional dependency will produce a
monomorphic \cd{Def} type.

The implementation function is required to produce a value of type \cd{Body r},
which is simply an Ivory monadic action with its allocation context hidden, and
return type exposed as the type variable \cd{r}.  The \cd{Body} type serves two
purposes: it removes the need to write an instance of \cd{IvoryProcDef} that
involves a rank-2 function, and it defines an extension point for modifying the
body of the procedure.  Pre- and post-conditions can be added to a procedure body
by the use of the \cd{requires} and \cd{ensures} functions, respectively.  Both
functions allow arbitrary Ivory statements to be added, but disallow all
effects.  The result of this restriction is that memory can be read and
validated, but control flow and allocation effects are prohibited.

The procedure body can be defined through the use of the \cd{body} function,
whose signature is shown in Figure~\ref{fig:proc-defs}, which lifts an Ivory
computation that returns a result \cd{r} and allocates data in a region \cd{s}
into a value of type \cd{Body r}.  As the allocation scope expected by the given
Ivory computation is quantified over in a rank-2 context by the \cd{body} function,
it \emph{cannot} appear in the type of the result, \cd{r}.  This
prevents anything allocated within the implementation function from
being returned, a source of dangling pointer bugs.  The same technique was
used by Launchbury and Peyton Jones~\cite{stmonad} to prevent mutable state
from leaking out of the context of the run function for the \cd{ST} monad.

For example, the procedure \cd{f} defined in Figure~\ref{fig:proc-def} will
produce a type error, as it attempts to return a locally-allocated reference;
references are parameterized by the scope they were allocated in, and as that
scope variable is quantified over in the rank-2 context of the argument to the
\cd{body} function, that reference is prevented from showing up in the return
type of the procedure, \cd{r}.

\begin{figure}[ht]
\begin{code}
f = proc "f" $ body $ do
  ref <- local (izero :: Init Sint32)
  ret ref
\end{code}
\caption{Attempted creation of a dangling pointer}
\label{fig:proc-def}
\end{figure}

\paragraph{Invocation} Procedures are through the use of the \cd{call} function,
which takes a \cd{Def} as its first argument, using its signature to determine
the arguments needed.  The arguments needed are determined by the \cd{IvoryCall}
class, which uses the signature information to produce a continuation that
requires parameters that match the type of the argument list from the signature
of the \cd{Def}.  The \cd{IvoryCall} class mirrors the structure of the
\cd{IvoryProcDef} instances structure, though it adds one additional parameter:
\cd{eff}.  This additional parameter is required so that the containing effect
context of the call can be connected to the result of the continuation generated
by the instances of \cd{IvoryCall}.  For example, calling a procedure with type
\cd{Def ('[Sint32] :-> Sint32)} will produce a continuation of the type,
\cd{Sint32 -> Ivory eff Sint32}, where the \cd{eff} parameter is inherited from
the current environment.

\begin{figure}[ht]
\begin{code}
data Proc k = [k] :-> k

class IvoryProcDef (sig :: Proc *) impl | impl -> sig
instance IvoryProcDef ('[] :-> r) (Body r)
instance IvoryProcDef (as :-> r) impl
  => IvoryProcDef ((a ': as) :-> r) (a -> impl)

class IvoryCall eff (sig :: Proc *) impl
  | sig eff -> impl, impl -> eff
instance IvoryCall eff ([] :-> r) (Ivory eff r)
instance (IvoryExpr a, IvoryCall eff (as :-> r) impl)
  => IvoryCall eff ((a ': as) :-> r) (a -> impl)

body :: (forall s. Ivory (ProcEffects s r) ())
     -> Body r

data Def (sig :: Proc *)
proc :: IvoryProcDef sig impl
     => Sym -> impl -> Def sig

call :: IvoryCall sig eff impl => Def sig -> impl
\end{code}
\caption{Function definition support.\eric{nitpicky: People like to skim through
  papers and just look at the figures, so the captions should be much more
  descriptive. Ideally, you could get the gist of the paper just be looking at
  the figures and reading the captions.}}
\label{fig:proc-defs}
\end{figure}


\subsection{Bit-Data}
\label{sec:bitdata}

\paragraph{Introduction}
Low-level systems programming often requires extensive manipulation
of binary data packed into multi-field integer values. For example,
a hardware register may contain several single-bit flags along with
multi-bit fields that may not be aligned to byte boundaries.

When programming in C, these bit values are typically accessed by
defining a set of integer constants and using bit operations to
shift and mask the correct bits into place with little to no type
safety.

In support of high assurance low-level programming, Ivory's standard
library contains a data definition language for these "bit data"
types. Our system is a subset of the bit data implementation
described in~\cite{high-level},
which allows the programmer to define bit data as algebraic data
types that can be nested and accessed in a type-safe manner.

\paragraph{Implementation}
Ivory's type system supports a set of unsigned integer types with
specific bit sizes (8, 16, 32, and 64 bits), as in the C language.
In order to support bit data of arbitrary width (up to the maximum
supported length of 64 bits), we use a type family
\cd{BitRep n} to map an integer size in bits to the smallest
concrete Ivory type that can hold an integer of that size:

\begin{code}
type family BitRep (n :: Nat) :: *
type instance BitRep 1 = Uint8
type instance BitRep 2 = Uint8
{- ... -}
type instance BitRep 64 = Uint64
\end{code}

Ivory adds additional type safety to arbitrary width integers by
wrapping these values in an opaque type \cd{Bits n}.
Haskell's module system is used to hide the raw constructor for
these values, only permitting valid values to be created via the
use of smart constructors:

\begin{code}
newtype Bits (n :: Nat) = Bits (BitRep n)

zeroBits :: Bits n
repToBits :: BitRep n -> Bits n
bitsToRep :: Bits n -> BitRep n
\end{code}

Smart constructors that are partial due to narrowing,
such as \cd{repToBits}, automatically mask out any bits that are
out of range. It is also possible to define runtime-checked versions
of these functions that treat such "junk values" as an error.

To support combining multiple bit fields into a single value, we
generalize the "bit data" concept with a type class \cd{BitData}
that captures the interface of a value that may be converted to
and from its representation as raw bits:

\begin{code}
class BitData a where
  type BitType a :: *
  toBits :: a -> BitType a
  fromBits :: BitType a -> a
\end{code}

The \cd{Bits n} type is a trivial instance of this type class:

\begin{code}
instance BitData (Bits n) where
  type BitType (Bits n) = Bits n
  toBits   = id
  fromBits = id
\end{code}

\paragraph{Type Definition}
The language for defining bit data types mirrors Haskell's syntax
for defining data types. Each bit data type contains one or more
constructors, each of which may have zero or more data fields.

For example, consider a control register for a communication device
with a 2-bit field used to specify the baud rate. We may define a
bit data type \cd{BaudRate} that enumerates the legal 2-bit values:

\begin{code}
[ivory|
  bitdata BaudRate :: Bits 2
    = baud_9600   as 0b00
    | baud_19200  as 0b01
    | baud_38400  as 0b10
    -- bit pattern 0b11 is invalid
|]
\end{code}

This definition defines an opaque Haskell type \cd{BaudRate},
implements an instance of the \cd{BitData} type class, and defines
zero-argument constructors for each value:

\begin{code}
newtype BaudRate = {- ... -}
instance BitData BaudRate where
  {- ... -}
baud_9600, baud_19200, baud_38400 :: BaudRate
\end{code}

Bit data types can be arbitrarily nested to define more complex types.
To continue the example, we define the entire control register consisting
of enable bits for a transmitter and receiver, along with the baud rate:

\begin{code}
[ivory|
  bitdata CtrlReg :: Bits 8 = ctrl_reg
    { ctrl_tx_enable  :: Bit
    , ctrl_rx_enable  :: Bit
    , ctrl_baud_rate  :: BaudRate
    } as 0b0000 # ctrl_tx_enable # ctrl_rx_enable
                # ctrl_baud_rate
|]
\end{code}

This definition of \cd{CtrlReg} defines a single constructor for building
a \cd{CtrlReg} value out of its constituent fields:

\begin{code}
ctrl_reg :: Bit -> Bit -> BaudRate -> CtrlReg
\end{code}

The field definitions define accessors for the fields of a
\cd{CtrlReg}. Because the type of these accessors contains both
the type of the containing bit data and the field being accessed,
Haskell's type system prevents errors such as accessing a bit
in the wrong register:

\begin{code}
ctrl_rx_enable :: BitDataField CtrlReg Bit
\end{code}

\paragraph{Usage}
In a typical low-level application, these fields
are accessed with a read-modify-write cycle which is supported
efficiently by the \cd{withBits} function and Haskell's
\cd{do} notation:

\begin{code}
init_ctrl_reg :: Def ('[] :-> ())
init_ctrl_reg = proc "init_ctrl_reg" $ body $ do
  reg <- call read_ctrl_reg
  call_ write_ctrl_reg $ withBits reg $ do
    setBit   ctrl_tx_enable
    setBit   ctrl_rx_enable
    setField ctrl_baud_rate baud_9600
\end{code}

\subsection{Module System}
\label{sec:modules}

As seen in Section~\ref{sec:ivory-overview}, Ivory's module system packages up
the collection of procedures, data declarations, and dependencies to be passed
to a back-end, such as the C code generator. The module system is implemented as
a writer monad that produces a list of abstract syntax values that are processed
by the various back-ends.

Because our primary backend is C, Ivory modules respect some of the conventions
of C modules in which header files are used to specify shared declarations. For
example, declarations can be declared as either public or private, and modules
can depend on other modules.

While Ivory's type system is embedded in GHC's type system and Ivory's language
is the Haskell term language, Ivory's module system cannot be embedded in
Haskell's module system. Thus, the user must deal with both Haskell's module
system and Ivory's orthogonal module system when programming.

At best, forgetting to include an Ivory dependency is an inconvenience. This
inconvenience can be substantial in the case that an inter-module dependency is
omitted, which still permits the Ivory program to type-check. If the dependency
missing is a C~function implementation, for example, C~code is generated and
compiles, but fails during link time. The error does not result in a safety
violation, but in large projects, such as the SMACCMPilot autopilot written in
Ivory~\cite{}, the error can take several minutes to detect.

\begin{figure}[ht]
\begin{code}
foo :: Def ('[Sint32] :-> Sint32)
foo = proc "foo" $ \_ -> body $ ret 0

fooInternal :: Def ('[Ref s (Stored Sint32)] :-> Sint32)
fooInternal = proc "foo" $ \ref -> body $ do
 x <- deref ref
 ret x

main :: Def ('[] :-> Sint32)
main = proc "main" $ body $ do
 x <- call foo 0
 ret x

cmodule :: Module
cmodule = package "Evil" $ do
 incl fooInternal
 incl main
\end{code}
\caption{Unsafe module usage}
\label{fig:unsafe-module}
\end{figure}

Worse, a naive implementation of the module system can lead to safety
violations. For example, consider the program in
Figure~\ref{fig:unsafe-module}. Two procedures, \cd{foo} and \cd{fooInternal}
are defined but given the same string used as the procedure name, used in the
generated C. The Ivory program is type-correct and safe, but by passing
\cd{fooInternal} into the module, it is compiled rather than \cd{foo}. And given
the C99 specification, the program compiles without warnings or errors, since
\cd{0} can be implicitly cast to a pointer to a signed 32-bit
integer.\footnote{In practice, the C we generate does contain a warning, since
  the C we generate contains additional type annotations.} The result is a
null-pointer dereference.

To ensure this does not happen, a simple type-check pass over the Ivory AST is
performed before compilation. The type-check pass ensures that the prototype of
a function matches the types of the arguments.


\newcommand{\coreivory}{Core Ivory}

\section{Ivory Semantics}
\label{sec:semantics}

In this section we discuss the more formal aspects of the Ivory
language, including its static and dynamic semantics.  We modeled a
simplified version of the Ivory language inside
Isabelle/HOL\sjw{cite}, henceforth \emph{\coreivory{}}; this
development is available under the \texttt{ivory-formal-model}
directory in the Ivory repository.  We present in this section a
semantics based upon the Isabelle/HOL development.

Developing This model provides a number of benefits for a modest
investment --- we developed the model in under a person month, albeit
one of the authors has significant experience with Isabelle.  In
addition to the basic benefits formalisation provides, we can
experiment with extensions to Ivory.

In one such experiment, we extended the model to allow references in
the heap, a feature we avoided in the development of Ivory due to
soundness concerns.  While a simple extension to the syntax and
semantics of Ivory, the effort involved in extending the soundness
proofs was almost as much as developing the initial model.  

\newcommand{\sep}{\ |\ }

\newcommand{\syntaxtitle}[1]{\multicolumn{3}{l}{\textit{#1}}}
\begin{figure}[t]

\[
\begin{array}{crl}
\syntaxtitle{pure expressions}\\
e & ::= & 0 \sep{} 1 \sep{} \ldots{} \sep{} \texttt{true} \sep{} \texttt{false} \sep{} () \sep{} x \sep{} 
          e_1 \mathbin{\mathit{op}} e_2 \\
\syntaxtitle{impure expressions}\\
i & ::= & \texttt{pure}(e) \sep{} \texttt{alloc}(e) \sep{}
          \texttt{read}(e) \sep{} \texttt{write}(e_1, e_2)\\
\syntaxtitle{statements}\\
s & ::= & \texttt{skip} \sep{} \texttt{return}(e) \sep{} s_1; s_2 \\ 
  & |   & \texttt{if}(e)\texttt{\;then\;} s_1 \texttt{\;else\;} s_2 \\
  & |   & \texttt{for}(x = e_1; e_2; e_3) \{ s \} \\
  & |   & \texttt{let\;} x = i \texttt{\;in\;} s \\
  & |   & \texttt{let\;} x = f(e_1, \ldots{}, e_n) \texttt{\;in\;} s \\
\syntaxtitle{procedure definitions}\\
P & ::= & \texttt{proc\;} f(x_1, \ldots{}, x_n) \{ s \}
\\
\syntaxtitle{values}\\
v & ::= & 0 \sep{} 1 \sep{} \ldots{} \sep{} \texttt{true} \sep{} \texttt{false} \sep{} () \\
w & ::= & \texttt{stored}(v) \sep{} \texttt{ref}(r, n) \\
\syntaxtitle{environments}\\
E & \in{} & x \to w \\
\syntaxtitle{regions and heaps}\\
R & \in{} & \mathbb{N} \to w\\
H & ::=   & H, R \sep{} \cdot{}\\
\syntaxtitle{stacks}\\
F & ::= & \texttt{rframe}(x, E, s) \sep{} \texttt{sframe}(E, s) \\
S & ::= & F, S \sep{} \cdot{} \\
\syntaxtitle{configurations}\\
C & \in & H \times S \times E \times s \sep{} \texttt{finished}(v) \\
\syntaxtitle{types}\\
\rho   & \in & \textit{region variables}\\
\alpha & ::= & \texttt{nat} \sep{} \texttt{bool} \sep{} \texttt{unit} \\
\tau   & ::= & \texttt{storedt}(\alpha) \sep{} \texttt{reft}(\rho, \alpha)
\end{array}
\]

\sjw{values (in or out?), $\in$ or $\subseteq$?}

\label{fig:syntax}
\caption{Concrete syntax of \coreivory{}}
\end{figure}

\sjw{more on why we want a model/motivations (?)}

\subsection{Syntax}

The syntax for \coreivory{} is given in \autoref{fig:syntax}.
\coreivory{} is based upon a typical typed imperative language with
function calls, references, and memory allocation (but not memory
deallocation).  \coreivory{} attempts to stay faithful to Ivory
wherever possible, and so variables are let-bound with forms for
binding the result of expression evaluation and function
calls. Furthermore, \coreivory{} expressions are stratified into
\emph{pure} and \emph{impure}, the latter allowing operations on the
heap: allocation, reading, and writing references.

Ivory uses regions to manage memory.  Thus, the heap is modeled as a
list of \emph{regions}, each region a finite map from \emph{offsets},
modeled as natural numbers, to \emph{stored values}; Ivory does not
allow references in heap allocated values, and so a stored value is
any value which is not a reference. A \emph{reference} contains both a
\emph{region index} into the list of regions, and an offset with the
region.  To simplify the presentation, we will use $H(r, n)$ to denote
the value at offset $n$ in the $r$th region of $H$, and similarly with
updates.

As with values, types classifying values in \coreivory{} are
stratified into storable and reference types; a reference type
$\texttt{reft}(\rho, \alpha)$ is a reference to an object of type
$\alpha$ in region $\rho$, where $\alpha$ is not a reference.

\sjw{talk about structs and arrays?}

\subsection{Operational Semantics}


\newcommand{\stepsX}[2]{\models #1 \longmapsto{} #2}
\newcommand{\stepsXX}[3]{\models #2 \longmapsto^{#1} #3}
\newcommand{\steps}[4]{\stepsX{#1; #2}{#3; #4}}

\newcommand{\denoteexp}[2]{\llbracket{}#1\rrbracket{}#2}

\newcommand{\stepsH}[2]{#1 \longmapsto_I #2}
\newcommand{\hsteps}[5]{#1 \models \stepsH{#2; #3}{#4; #5}}

\newcommand{\wfstmt}[5]{#1; #2; #3 \vdash_s #4 : #5}
\newcommand{\wfexp}[3]{#1 \vdash_e #2 : #3}
\newcommand{\wfimp}[4]{#1; #2 \vdash_i #3 : #4}

\begin{figure*}[ht]

\begin{mathpar}

%% Pure expressions

\sjw{all but var lookup return stored(x)}
\denoteexp{e}{E} = \ldots

\and
%% Impure expression

\inferrule{\denoteexp{e}{E} = w}{\hsteps{E}{H}{\texttt{pure}(e)}{H}{w}}
\and
\inferrule{\denoteexp{e}{E} = \texttt{stored}(v)}{\hsteps{E}{H, R}{\texttt{alloc}(e)}{H, R[p \mapsto{} v]}{\texttt{ref}(|H|, p)}}(p \notin{} \mathrm{dom}(R))
\and
\inferrule{\denoteexp{e}{E} = \texttt{ref}(r, n)}{ \hsteps{E}{H}{ \texttt{read}(e) }{H}{\texttt{stored}(H(r, n))} }
\and
\inferrule{\denoteexp{e_1}{E} = \texttt{ref}(r, n) \\ \denoteexp{e_2}{E} = \texttt{stored}(v)}{ \hsteps{E}{H}{ \texttt{write}(e_1, e_2) }{H[(r, n) \mapsto v]}{\texttt{stored}(())} }

\\
%% Statements

\inferrule{ {} }{\steps{(H; \texttt{sframe}(E, s), S; \_)}{\texttt{skip}}{(H; S; E)}{s}}
\and
\inferrule{ {} }{\steps{(H; \texttt{sframe}(\_, \_), S; E)}{\texttt{return}(e)}{(H; S; E)}{\texttt{return}(e)}}
\and
\inferrule{ \denoteexp{e}{E_0} = w }{\steps{(H, \_; \texttt{rframe}(x, E, s), S; E_0)}{\texttt{return}(e)}{(H; S; E[x \mapsto{} w])}{s}}
\and
\inferrule{ \denoteexp{e}{E} = w }{ \stepsX{(\_; \cdot; E); \texttt{return}(e)}{\texttt{finished}(w)} }
\and
\inferrule{ {} }{\steps{(H; S; E)}{ (s_1; s_2) }{(H; \texttt{sframe}(E, s_2), S; E)}{s_1}}
\and

\inferrule{ \denoteexp{e}{E} = \texttt{stored}(\texttt{true}) }{\steps{(H; S; E)}{ \texttt{if}(e)\texttt{\;then\;} s_1 \texttt{\;else\;} s_2 }{(H; S; E)}{s_1}}
\and
\inferrule{ \denoteexp{e}{E} = \texttt{stored}(\texttt{false}) }{\steps{(H; S; E)}{ \texttt{if}(e)\texttt{\;then\;} s_1 \texttt{\;else\;} s_2 }{(H; S; E)}{s_2}}
\and
\inferrule{ \denoteexp{e_1}{E} = w }{\steps{(H; S; E)}{ \texttt{for}(x = e_1; e_2; e_3) \{ s \} }{(H, S, E[x \mapsto{} w])}%
{\texttt{if}(e_2)\texttt{\;then\;} (s; \texttt{for}(x = e_3; e_2; e_3) \{ s \}) \texttt{\;else\;} \texttt{skip} }}
\and
\inferrule{ \hsteps{E}{H}{i}{H'}{w} }{\steps{(H; S; E)}{\texttt{let\;} x = i \texttt{\;in\;} s}{(H'; S; E[x \mapsto{} w])}{s}}
\and
\inferrule{ \denoteexp{e_i}{E} = w_i \\ \texttt{proc\;} f(x_1, \ldots, x_n)\{ \mathit{body} \} \in \texttt{Procs} }{\steps{(H; S; E)}{\texttt{let\;} x = f(e_1, \ldots{}, e_n) \texttt{\;in\;} s}{(H, \emptyset; \texttt{rframe}(x, E, s), S; [x_1 \mapsto{} w_1. \ldots{}, x_n \mapsto{} w_n])}{\mathit{body}}}

\\
% wfimp
\inferrule{ \wfexp{\Gamma}{e}{\tau} }{ \wfimp{\Gamma}{\rho}{\texttt{pure}(e)}{\tau} }b
\and
\inferrule{ \wfexp{\Gamma}{e}{\alpha} }{ \wfimp{\Gamma}{\rho}{\texttt{alloc}(e)}{\texttt{reft}(\rho, \alpha)} }
\and
\inferrule{ \wfexp{\Gamma}{e}{\texttt{reft}(\rho, \alpha)} }{ \wfimp{\Gamma}{\rho}{\texttt{read}(e)}{\texttt{storedt}(\alpha)} }
\and
\inferrule{ \wfexp{\Gamma}{e_1}{\texttt{reft}(\rho, \alpha)} \\ \wfexp{\Gamma}{e_2}{\alpha} }{ \wfimp{\Gamma}{\rho}{\texttt{write}(e_1, e_2)}{\texttt{storedt}(\texttt{unit})} }

\\
%% wfstmt
\inferrule{ }{ \wfstmt{\Gamma}{\Psi}{\rho}{\texttt{skip}}{\tau} }
\and
\inferrule{ \wfexp{\Gamma}{e}{\tau}  }{ \wfstmt{\Gamma}{\Psi}{\rho}{\texttt{return}(e)}{\tau} }
\and
\inferrule{ \wfstmt{\Gamma}{\Psi}{\rho}{s_1}{\tau} \\ \wfstmt{\Gamma}{\Psi}{\rho}{s_2}{\tau} }{ \wfstmt{\Gamma}{\Psi}{\rho}{s_1; s_2}{\tau} }
\and
\inferrule{ \wfexp{\Gamma}{e}{\texttt{bool}} \\ \wfstmt{\Gamma}{\Psi}{\rho}{s_1}{\tau} \\ \wfstmt{\Gamma}{\Psi}{\rho}{s_2}{\tau} }%
{ \wfstmt{\Gamma}{\Psi}{\rho}{\texttt{if}(e)\texttt{\;then\;} s_1 \texttt{\;else\;} s_2 }{\tau} }
\and
\inferrule{ \wfexp{\Gamma}{e_1}{\sigma} \\ \wfexp{\Gamma[x \mapsto \sigma]}{e_2}{\texttt{bool}} \\ \wfexp{\Gamma[x \mapsto \sigma]}{e_3}{\sigma}
\\ \wfstmt{\Gamma[x \mapsto \sigma]}{\Psi}{\rho}{s}{\tau}}%
{ \wfstmt{\Gamma}{\Psi}{\rho}{ \texttt{for}(x = e_1; e_2; e_3) \{ s \} }{\tau} }
\and
\inferrule{ \wfimp{\Gamma}{\rho}{i}{\sigma} \\  \wfstmt{\Gamma[x \mapsto \sigma]}{\Psi}{\rho}{s}{\tau} }%
{ \wfstmt{\Gamma}{\Psi}{\rho}{\texttt{let\;} x = i \texttt{\;in\;} s}{\tau} }
\and
\inferrule{ \wfexp{\Gamma}{e_i}{\sigma_i} \\
            \Psi(f) = \texttt{fun}(\sigma, (\sigma_1, \ldots, \sigma_n))\\
            \wfstmt{\Gamma[x \mapsto \sigma]}{\Psi}{\rho}{s}{\tau} }%
{ \wfstmt{\Gamma}{\Psi}{\rho}{\texttt{let\;} x = f(e_1, \ldots{}, e_n) \texttt{\;in\;} s}{\tau} }

\\
\inferrule{ \Psi(f) = \texttt{fun}(\tau, (\sigma_1, \ldots, \sigma_n))\\            
            \texttt{proc\;} f(x_1, \ldots, x_n)\{ \mathit{s} \} \in \texttt{Procs} \\            
            \texttt{frees}(\tau) \subseteq \texttt{frees}(\sigma_1) \cup \ldots \cup \texttt{frees}(\sigma_n) \\
            \wfstmt{[x_1 \mapsto \sigma_1, \ldots, x_n \mapsto \sigma_n]}{\Psi}{\rho}{s}{\tau} }%
{ \vdash \texttt{Procs} : \Psi }(\textrm{for all }f\textrm{ in }\textrm{dom}(\Psi), \rho \textrm{ fresh})

\end{mathpar}

\label{fig:rules}
\caption{Selected semantic and typing rules. The semantics and typing rules for expressions ($\denoteexp{e}{E}$ and $\wfexp{\Gamma}{e}{\tau}$) are standard and are so omitted.}
\end{figure*}

\sjw{mention missing rules in the figure} \coreivory{}'s semantics,
presented in \autoref{fig:rules}, is modeled as an abstract
machine\sjw{cite?} over configurations.  The judgement
\[
\stepsX{C}{C'}
\]
states that configuration $C$ transitions to configuration $C'$ and
relies on the denotation of expressions (not shown) $\denoteexp{e}{E}$
and the semantics of impure expressions
\[
\hsteps{E}{H}{i}{H'}{w}
\]
stating that execution of the impure expression $i$ in environment $E$
and heap $H$ yields a new heap $H'$ and value $w$.

A configuration consists of a heap, a stack, an environment, and a
statement. The stack contains continuations for both function calls
and statement sequences; the environment maps variables to values. The
semantics of sequencing is slightly non-standard as variables are
let-bound rather than assigned, and so statement sequencing preserves
the environment across execution of the first statement.

Operationally, the heap is extended on a function call, an empty
region being added to the end of the list, and shrunk on function
return, removing the last region.  Allocating an object extends the
current (last) region.

A configuration is stuck if there is no available transition.  For
instance, an attempted heap access or update where the region index
does not exist or at an offset which has not been allocated will
result in a stuck configuration.  In particular, accessing a region
after it has been removed will result in a stuck state.  

\sjw{make stuc states more explicit?}
% \sjw{stratification of values into ref/non-ref makes things easy}

\subsection{Typing Ivory}

The typing rules for \coreivory{} statements are given in
\autoref{fig:rules}.  The typing judgement 
\[
\wfstmt{\Gamma}{\Psi}{\rho}{s}{\tau}
\]
holds when the statement $s$ is well-formed under the variable
environment $\Gamma$, procedure environment $\Psi$, current region
$\rho$, with any return statements returning values of type $\tau$.

The region variable $\rho$ represents the current region and is used
when checking memory allocation.  In particular, we may derive the
following typing rule
\[
\inferrule{ 
\wfexp{\Gamma}{e}{\alpha} \\
\wfstmt{\Gamma[x \mapsto \texttt{reft}(\rho, \alpha)]}{\Psi}{\rho}{s}{\tau} }%
{ \wfstmt{\Gamma}{\Psi}{\rho}{\texttt{let\;} x = \texttt{alloc}(e) \texttt{\;in\;} s}{\tau} }
\]
where the body of the let statement is checked under the additional
assumption that the variable $x$ has reference type
$\texttt{reft}(\rho, \alpha)$.  

The typing rule for procedure bodies ensures that this region variable
is fresh; this constraint, together with the constraint that region
variables in the procedure's return type must occur in an argument
type, ensures that references cannot escape the scope in which they
were allocated.  These constraints are then fundamental to the type
safety of \coreivory{} programs.

\subsection{Type Safety}

We prove type safety, that is, well-typed programs do not get stuck,
by proving the usual progress and preservation lemmas.  As is
common\sjw{cite?}  with type safety for imperative languages, we
define auxiliary well-formedness invariants on configurations.  The
progress lemma then states that well-formed configurations are not
stuck, and preservation states that well-formed configurations
transition to well-formed configurations.

\sjw{name this?}
Formally, we prove, assuming $\vdash \texttt{Procs} : \Psi$ and
$\Psi(\texttt{main}) = \texttt{fun}(\texttt{nat}, ())$ then, for some
number of steps $n$ and $v \in \mathbb{N}$, 
\[
\stepsXX{n}{(\cdot; \cdot; \cdot); \texttt{let\;} x = \texttt{main}() \texttt{\;in\;} \texttt{return}(x)}{\texttt{finished}(v)}
\]
or, for all $n$ there is a well-formed configuration $C$ such that
\[
\stepsXX{n}{(\cdot; \cdot; \cdot); \texttt{let\;} x = \texttt{main}() \texttt{\;in\;} \texttt{return}(x)}{C}
\]
Informally, a well-formed program, when called via the \texttt{main}
procedure, will either terminate in a finite number of steps, or will
diverge through well-formed configurations.

The well-formedness invariants are typical, and are based upon a
well-formed value judgement.  For our purposes, the interesting rule
here is for references
\[
\inferrule{\Delta(\rho) = r \\ \Theta(r, n) = \alpha }%
{\Theta; \Delta \vdash \texttt{ref}(r, n) : \texttt{reft}(\rho, \alpha)}
\]
which links reference types to reference values through the region
environment $\Delta$, mapping region variables to region indices, and
heap type $\Theta$, mapping region indices and offsets to types.
Well-formed heaps and environments then follow point-wise from this
judgement.  Well-formedness of stacks follows, ensuring that each
continuation on the stack is well-formed.

A well-formed configuration, in addition to the well-formedness of the
heap, stack, variable environment, and current statement, constrains
the region environment ($\Delta$).  In particular, the variable
representing the current region must be mapped to the length of the
current heap, every region index in the range of $\Delta$ must be
below this length, and the type variables occurring in the variable
environment $\Gamma$ must be mapped by $\Delta$.

The progress lemma follows from well-formedness.  The preservation
proof, as usual, is the trickier of the two proofs.  In particular,
the case for return involves showing that the various configuration
members are well-formed under a heap where the last region has been
removed from the heap.  This involves showing that references are
well-formed under this smaller heap which follows from the stack and
region environment well-formedness invariants.  The case for function
calls is also involved, although this is primarily due to the
instantiation of the type variables in the type of the called
function.

The type safety proofs are greatly simplified by the restriction of
heap values to non-references: it is trivially true, for example, that
the heap is well-formed after a return as the well-formedness of
non-reference values does not depend on the type of other heap
elements.  In the next section we discuss the relaxation of this
restriction.

\subsection{Nested Regions}

\begin{itemize}
\item Simpler syntax (no stratification)
\item Much more complex proofs/invariants
\item What are the main ideas (downward closed heap)?
\end{itemize}

\subsection{Discussion}

\begin{itemize}
\item modeling gap
\item various bugs in Ivory but not in Isabelle model
\item time estimate (approx 1 month?)
\item future work: first-class regions?
\item what have we gained?
\end{itemize}











% \lee{Simon: this is yours. Give the typing rules for Ivory, sketch the proofs of
%   progress and preservation, and describe this embedding in Isabelle. Also
%   describe the (work-in-progress) to extend Ivory with memory regions and
%   relaxed references. I'm expecting this'll be about 1-2 pages. Talk about a
%   little additional type-checking at value level (return statements), discovered
% from formal modeling.}


\section{Ivory Testing and Verification}
\label{sec:tools}

Ivory contains built-in tools to support high-assurance software
development. These tools include a correctness condition generator, a SMT-based
symbolic simulator, a theorem-prover translator, and a QuickCheck engine. We
describe each of them briefly below.

None of these tools specifically depend on Ivory being an EDSL, and their
implementations are straight-forward.

\subsection{Correctness Conditions}
Some correctness properties (e.g., arithmetic underflow and overflow properties)
cannot be embedded in the Haskell type system. For these kind of properties, the
Ivory AST is instrumented with appropriate assertions. For example, for the
function

\begin{code}
main :: Def ('[Sint32, Sint32] :-> Sint32)
main = proc "main" $ \ x y -> body $
  ret (x + y)
\end{code}
\noindent
The following function is generated, which contains an overflow check before the
expression is evaluated:
\begin{code}
int32_t main(int32_t var0, int32_t var1)
{
    bool i_ovf0 = add_ovf_i32(var0, var1);
    COMPILER_ASSERTS(i_ovf0);
    return (int32_t) (var0 + var1);
}
\end{code}
\noindent
The overflow check is performed by the function
\cd{add\_ovf\_i32}. \cd{COMPILER\_ASSERTS} is a C macro; the appropriate response when
property is violated is platform-dependent (e.g., logging, do-nothing, run a
recovery procedure, etc.).

Ivory inserts correctness condition checks for the following, as requested by
the user:
\begin{itemize}
\item no arithmetic underflow and overflow,
\item no division-by-zero,
\item no bit-shifts are greater than or equal to the value's width,
  (bit-shifts on signed integers is prevented statically by the type system)
\item no floating-point operations result in \cd{inf} or \cd{NaN} values.
\end{itemize}

The implementation is mostly straightforward; the two aspects that require some
care of are sharing and control-flow expressions. As an EDSL, Ivory encourages
the use of macros which can result in large expressions. In this case, each
subexpression must be checked. Standard common subexpression elimination can
dramatically reduce the number of instrumented assertions. Second, Ivory contains
the short-cutting expressions of conjunctions, disjunctions, and conditionals,
analogs of C's \cd{\&\&}, \cd{||}, and \cd{\_ ? \_:\_}, respectively. For these
expressions, the generated assertions must contain as a precondition that
short-cutting has not occurred so as not to be overly pessimistic. For example,
for the expression
\noindent
\begin{code}
x != 0 ? 3/x : 0
\end{code}
\noindent
the assertion
\begin{code}
(x == 0) || (x != 0)
\end{code}
\noindent
is generated (and in this case, subsequently discharged by the constant folder).

Generated correctness conditions as well as general user assertions can be
discharged via testing or using model-checking or theorem-proving. We describe
Ivory's tooling for these approaches below.

\paragraph{Symbolic Simulation}
Ivory contains a symbolic simulator built over CVC4~\cite{cvc4} for verifying
programs. Ivory programs are typically amenable to formal analysis since they
guarantee the absence of memory-safety errors, there is no pointer arithmetic,
there is no heap, and they are not concurrent. Given a set of preconditions, the
simulator attempts to verify any inline assertions and postconditions. The
simulator ensures that the collection of preconditions are satisfiable.

The symbolic simulator abstracts various domains. Floating point types are
abstracted as reals. However, fixed-width values are modeled precisely as are
arrays. We have not yet incorporated a theory of bit-vectors yet,
however. \eric{The last two sentences may be a bit confusing, since you'd usually expect a precise encoding of fixed-width integers to use bit-vectors. Of course we get away with using integers since we strictly disallow overflow. Perhaps we should clarify here?} Non-linear operators are abstracted with linear contracts. Loops are
unrolled (all loops in Ivory have a constant upper bound).

During analysis, inter-procedural calls are inlined or abstracted based a on
user-supplied option. If they are abstracted, then the callee's precondition is
added as a verification condition at the call site. The callee's postcondition
is added to the set of invariants following the call. For imported procedures
hand-written in C, Ivory allows them to be augmented with pre- and
postconditions to abstract them.

The simulator is simple, providing only bounded model-checking, no support for
concurrency, and no counter-example guided refinement, matching the simplicity
of the language.

We have used the symbolic simulator to analyze various Ivory programs, having
found an off-by-one bug in a ring buffer and verifying the correctness of a
safety state-machine in SMACCMPilot~\cite{smaccm}.

\paragraph{Theorem Proving}

Eakman~et~al. has implemented a theorem-prover back-end for
Ivory~\cite{ivory-acl2} that targets ACL2~\cite{acl}. The theorem-prover
performs inter-procedural analysis, abstracting procedure calls by their
contracts, like the symbolic simulator. However, the theorem prover has the
ability to scale more substantially, at the price of doing interactive
proofs. Eakman~et~al. provide examples of the use of the theorem-proving
back-end.

\paragraph{Property-Based Testing}

Finally, we have implemented a QuickCheck~\cite{qc} like property-based testing
framework for Ivory. The framework tests procedures by randomly generating
values both for their formal arguments as well as for global values referenced
by the procedure. Ivory is primarily a compiled language, so tests are compiled
to C to be executed. Only values that satisfy procedure preconditions should be
generated, so the framework contains an interpreter for evaluating values before
code generation.




\section{A Safe-C EDSL}
\label{sec:edsl}

Large-scale, safe~C programming in an EDSL has advantages and disadvantages. In
a previous experience report, we explored some of the benefits of using an EDSL
for embedded system development~\cite{smaccm}. We will not repeat those claims
here. Rather, we describe two problematic aspects we have specifically addressed in Ivory:
(1) integrating a C-like concrete syntax into the EDSL and (2) error-reporting.

\paragraph{Concrete Syntax}

A benefit of the EDSL approach is that it relieves the developer from having to
define and implement a front-end syntax. However, that also generally means that
only users of the host language will be attracted to using the
EDSL. We want C/C++ developers to use Ivory!

\begin{figure}[h!]
\begin{code}
void mapProc(s*uint8_t[4] arr, uint8_t x) {
  map ix {
    let v = arr@ix;
    *v = *v + x;
  }
}
\end{code}
  \caption{Concrete Syntax for Ivory}
  \label{fig:concrete}
\end{figure}

The need for a concrete C-like syntax became evident in our work supporting
Boeing's use of Ivory, mentioned in Section~\ref{sec:introduction}. We have
already seen specific uses of quasi-quotation in Ivory to define a concrete
syntax for structs (Section~\ref{sec:area}) and bit-data
(Section~\ref{sec:bitdata}). In fact, a quasi-quotation is given for the entire
Ivory language. An example quasi-quoted Ivory procedure is shown in
Figure~\ref{sec:control}. The procedure is the equivalent of the procedure with
the same name in Section~\ref{sec:control}.

A quasi-quoted Ivory program is guaranteed to be type-safe, since the generated
Haskell program is type-checked. an important feature of the quasi-quoted
language is that it automatically generates the appropriate type signatures for
Ivory programs, relieving the programmer from doing so. The quasi-quoter also
generates Ivory modules automatically and guarantees that procedure names match
their Haskell identifiers, obviating the problems discussed in
Section~\ref{sec:modules}. The quasi-quoter supports anti-quotation, so that
Haskell can still be used as a macro language. All of Boeing's development in
Ivory is via the quasi-quoter.

Implementing a quasi-quoter means that we have defined a lexer and parser for
the concrete syntax. Because Ivory is deeply embedded in Haskell, there is a
concrete data type (i.e., the abstract syntax tree (AST)) over which
optimizations and back-ends are implemented. The distance from Ivory as an EDSL
and a stand-alone compiler is surprisingly small, essentially requiring a
type-checker and a front-end targeting the AST directly. In this manner, we are
able to ``grow'' a compiler, from an EDSL to a stand-alone system.

\paragraph{Error Reporting}
Ivory's The use of advanced type-system features can produce elaborate and
confusing type-error messages to the user. Idris~\cite{christiansen2014reflect}
allows the programmer to supply error handlers, which are given a data structure
representing the compiler's error message and can rewrite it to insert
domain-specific knowledge. Unfortunately, GHC Haskell does not currently support
such a feature.

%% Thus, the EDSL author can provide custom type-error
%% messages using language the application programmers will understand.

The next best thing is to at least give the user accurate error location
information. Dynamically reporting errors in an EDSL is also made difficult by the lack of
source locations. Consider the Ivory expression \hbox{\cd{x / y},} which induces a
runtime assertion \cd{y != 0}. If the assertion fails, we would like to include
a \emph{Haskell} source location in the error message to direct the programmer to the source of
the error. But how are we to obtain the source location in the first place?
Ivory programs are comprised of Haskell expressions and Haskell is
pure--functions cannot depend on their call-site--so we must look outside the
language proper to obtain source locations. A common approach for writing
location-aware functions in Haskell, epitomized by the \cd{file-location}
package~\cite{file-location}, is to use a Template Haskell splice to query the
compiler for the current source location and insert it into the AST. This
approach did not appeal to us as Template Haskell incurs considerable syntactic
overhead, and we already had 10+ KLoC of Ivory which would all need to be
rewritten.

Instead we opted to write a compiler plugin that rewrites GHC's intermediate
representation to add the source locations. The implementation was
straightforward, we first extended the \cd{Ivory} statement type with a new
\cd{Location} constructor that just contains a location in the Haskell source,
essentially a special type of comment. The plugin then extracts the source
locations from GHC and wraps all actions in the \cd{Ivory} monad with a
\cd{withLocation} function that emits a \cd{Location} statement before executing
the wrapped action. While this effectively limits the location granularity to
lines rather than columns, our approach required only modest changes to the
Ivory compiler and--importantly--no changes to Ivory code. As the plugin
itself requires little knowledge of Ivory, we have abstracted it out into a
separate package~\cite{ghc-srcspan-plugin} that can be reused by other projects.
Furthermore, we have since submitted a lightweight extension to GHC that allows
functions to request their call-site by taking a special implicit
parameter~\cite{lewis2000implicit} as an argument, which should become available
in the 7.12 release.

\lee{Lee: describe line number insertions (overcoming a bad thing), module
  system (hack), the use of TH to make a concrete syntax, macros (e.g., standard
  lib---compare to Rust), other uses of macros, small compiler.}


\section{Related Work}
\label{sec:related-work}

The general idea of safe C languages is not new; our main contribution is
embedding a type system into GHC as well as our support of verification
tools.

Pioneering work in the area is the Cyclone language and
compiler~\cite{cyclone}. Cyclone is a dialect of C. Cyclone is less restrictive than
Ivory, relying on both static analysis and runtime checks to enforce memory
safety. Cyclone provides regions for dynamic memory allocation; garbage
collection is optional. Cyclone programs are typically slightly larger than
their C equivalents and mostly syntactically the same. In contrast to Ivory,
Cyclone does not provide macro-programming facilities (beyond the C
preprocessor), nor does it interface to verification and testing
tools. Unfortunately, Cyclone is not actively maintained.

Bit-data and memory areas in Ivory borrow heavily from Diatchki~et~al.'s
previous work~\cite{high-level, memareas}. Indeed, one can consider the present
work as demonstrating the feasibility of embedding this language into Haskell
and GHC types. BitC is another deprecated research language that explored a
similar design space~\cite{bitc}.

Spark/Ada is a mature language for high-assurance embedded
programming, with a contract language and verification tools to
prove invariants~\cite{spark}. To support verification, the language is
very restrictive: in particular, there are no references in the language.

Rust is an actively-developed safe C language, originating from
Mozilla~\cite{rust}. Rust has a powerful type system to enforce the safe use of
heap-based data structures. An affine type system prevents pointer aliasing
errors. Rust provides reference counting garbage collection as a library (so
other garbage collection strategies can be used instead). Rust has hygenic
macros. The language has a property-based testing framework, but no mature
support for verification, or even static checks for undefined behavior (e.g.,
division by a constant zero expression).

EDSLs for safe C programming also exist. Examples include Atom, a language for
lock-free embedded programs~\cite{atom}; Copilot, a stream-oriented synchronous
language~\cite{copilot}; SBV, a Haskell-based SMT symbolic simulator with a C code
generator~\cite{sbv}; and Feldspar, a language specialized for high-level and
efficient specifications of digital signal processing~\cite{feldspar1}. Compared to these
languages, Ivory is more focused on the kinds of C in low-level code, such as
device drivers, with bit-data and memory area manipulation.


\section{Conclusion}
\label{sec:conlusion}

\lee{Ivory, yay.}

\lee{Other approaches to a module system}

\lee{unions}



\lee{summarize the TCs over the AST?}

%% \appendix
%% \section{Appendix Title}



\acks

This work is supported by DARPA under contract no. FA8750-12-9-0169.  Opinions
expressed herein are our own.

% We recommend abbrvnat bibliography style.

\bibliographystyle{abbrvnat}
\bibliography{paper}

% The bibliography should be embedded for final submission.

%% \begin{thebibliography}{}
%% \softraggedright

%% \bibitem[Smith et~al.(2009)Smith, Jones]{smith02}
%% P. Q. Smith, and X. Y. Jones. ...reference text...

%% \end{thebibliography}


\end{document}

%                       Revision History
%                       -------- -------
%  Date         Person  Ver.    Change
%  ----         ------  ----    ------

%  2013.06.29   TU      0.1--4  comments on permission/copyright notices

