\section{Ivory Embedding}
\label{sec:ivory-embedding}

\lee{This section will be the heart of the paper, focusing on how we have
  embedded Ivory in Haskell's type system. I expect this to be 3-5 pages and
  probably the most technical. We want to be detailed, but describe things
  general enough that ML, Agda, etc. programmers can understand.
}

\subsection{The Ivory Monad}

\lee{Lee: describe the Ivory monad (not too much to say), then describe
  effects. Make sure to make clear these are monad effects like are popular in
  the literature. Mention inspiration from
  \url{http://www.doc.ic.ac.uk/~wlj05/files/Deconstraining.pdf}, but didn't
  follow implementation.}

\lee{Also talk about the use of typeclasses \cd{IvoryVar}, \cd{IvoryExpr},
  etc. and how this allows the user to define custom data, safely.}

Ivory is a monadic language, in which Ivory statements have effects in the Ivory
monad. The Ivory monad has the type

\begin{code}
Ivory (eff :: Effects) a
\end{code}

\noindent
and is a wrapper for a writer monad transformer over a state monad. The writer
monad writes statements into the Ivory abstract syntax tree, and the
state monad is used to generate fresh names for variables.

\paragraph{Effects}
The \cd{eff} type parameter of the Ivory monad is a phantom type that tracks
effects in the Ivory monad at the type level. (These effects have no relation to
the recent work on effects systems for monad transformers~\cite{}.) Currently,
we track three classes of effects for a given monadic code block:

\begin{itemize}
\item \emph{Returns}: does the code block contain a \cd{return} statement?
\item \emph{Breaks}: does the code block contain a \cd{break} statement?
\item \emph{Allocation}: does the code block contain local memory allocation?
\end{itemize}

Intuitively, these effects matter because their safety depends on the context in
which the monad is used. For example, a \cd{return} statement is safe when used
within a procedure, to implement a function return. But an Ivory code block can
also be used to implement an operating system task that should never
return. Similarly in Ivory, \cd{break} statements are used to terminate
execution of an enclosing loop (the other valid use of \cd{break} in C99 is to
terminate execution in a \cd{switch} block; Ivory does not contain
\cd{switch}). By tracking break effects, we can ensure that an Ivory block
containing a \cd{break} statement is not used outside of a loop. Finally,
allocation effects are used to guarantee that a reference to locally-allocated
memory is not returned by a procedure, resulting in undefined behavior; see
Section~\ref{sec:ref} for details. Moreover, from a code block's type alone, we
can determine whether it allocates memory, simplifying tasks like stack usage
analysis.

The Ivory effects system is essentially implemented by a type-level tuple
structure where each of the three effects correspond to a field of the
tuple. Type equality constrains enforce that a particular effect is (or is not)
allowed in a give function signature.

The type-level tuple is implemented using data kinds~\cite{}. Effects have the
kind \cd{Effects}, containing a single type constructor, \cd{Effects}. The type
constructor \cd{Effects} is parameterized by the fields of the tuple,
representing the respective effects. Using GHC's data kinds extension to lift
data type declarations to data kind declarations,
\begin{code}
data Effects = Effects ReturnEff BreakEff AllocEff
\end{code}

\noindent
Consider the return effects type, \cd{ReturnEff}.  Again using the data kinds
extension to define a new kind and its types, we define two types denoting
whether returns are permitted or not.

\begin{code}
data ReturnEff = forall t. Returns t | NoReturn
\end{code}

\noindent
The type is existentially quantified, since in the case of permitted returns, we
parameterize the type constructor with the type of the value being returned.

A type family~\cite{} is used to access and modify the types at each field of
the tuple. To access that field, a type family rewrites the \cd{Effects} type to
the return effect type \cd{ReturnEff}.

\begin{code}
type family   GetReturn (effs :: Effects) :: ReturnEff
type instance GetReturn ('Effects r b a) = r
\end{code}

\noindent
(The GHC convention is to precede a type with a tick (\cd{'}) to disambiguate a
type constructor (with the tick) from the data constructor.)

To modify the field, another type family rewrites it. For example,

\begin{code}
type family   ClearReturn (effs :: Effects) :: Effects
type instance ClearReturn ('Effects r b a) =
  'Effects 'NoReturn b a
\end{code}

With this machinery, we can now use a type equality constraint to enforce
particular effects in a context. For example, the type of \cd{ret}, the smart
constructor for returning a value from a procedure, has the following type
(additional type constraints are elided):

\begin{code}
ret :: (GetReturn eff ~ Returns r, ...)
    => r -> Ivory eff ()
\end{code}


\subsection{References and Allocation}
\label{sec:ref}
Ivory manages allocated data through the use of non-nullable references.
References are represented using the \cd{Ref} type, which takes two parameters:
the scope that it was allocated in, and the type of the memory area it points
to.

All references in Ivory are allocated in one of two scopes, \cd{Global}, or a
fresh local scope, unique to the function that the allocation takes place
within.  References with \cd{Global} scope are allocated through the use of the
\cd{area} top-level declaration, and references allocated within a function are
allocated through the use of the \cd{local} function.  Both can take
initializers, defaulting to zero-initialization when they are omitted.

While references allocated in the \cd{Global} scope are allocated when the
program starts, references allocated in the local scope of a function are stack
allocated, and freed as soon as the function returns.

\subsection{Memory Areas}
In Ivory, data pointed to by references and pointers is described by the {\tt
Area} kind, following \cite{memareas}.  This kind contains four types: {\tt
Stored}, \cd{Struct}, \cd{Array}, and \cd{CArray}.

\begin{figure}[h]
\begin{code}
data Area k = Array Nat (Area k)
            | CArray (Area k)
            | Struct Symbol
            | Stored k
\end{code}
\caption{The definition of the \cd{Area} kind}
\end{figure}

\paragraph{Stored values}
The simplest type of memory area is a single base type, lifted to the \cd{Area}
kind by the use of the \cd{Stored} type constructor.  For example, the type of
a single stored \cd{Sint32} would simply be \cd{Stored Sint32}.

\paragraph{Structs}
The struct type identifies an area of memory as being structured by the
corresponding struct declaration from a use of the ivory quasiquoter.  For
example, if a region of memory is typed using the struct definition from figure
\ref{example-struct}, it would have type \cd{Struct "a"}.

\begin{figure}[h]
\begin{code}
[ivory| struct a { field1 :: Stored Sint32
                 , field2 :: Struct "b"
                 }
      |]
\end{code}
\caption{An example struct definition}
\label{example-struct}
\end{figure}

Also introduced by the struct declaration are field labels.  Field labels allow
for indexing into a memory area, producing a reference to the value contained
within the struct.  For example, using the struct definition from figure
\ref{example-struct}, there would be two labels introduced: \cd{field1}, and
\cd{field2}, and will have types \cd{Label "a" (Stored Sint32)} and \cd{Label
"a" (Struct "b")}, respectively.

Using a struct label to select the field of a structure requires the use of the
\cd{(\mytilde>)} operator, which expects a reference to a structure as its first
argument, and a compatible label as its second.  In figure
\ref{example-struct-label}, the \cd{(\mytilde>)} operator is used with a reference to
an ``a'' struct, with the \cd{field1} label, producing a new reference of type
\cd{Ref Global (Stored Sint32)}

\begin{figure}[h]
\begin{code}
example :: Ref Global (Struct "a")
        -> Ref Global (Stored Sint32)
example ref = ref \mytilde> field1
\end{code}
\caption{Struct field indexing}
\label{example-struct-label}
\end{figure}

\paragraph{Arrays}
Arrays in Ivory take two type parameters: the length of the array as a
type-level natural number, and the area type of its elements.  For example, an
array of 10 \cd{Sint32} would have the type \cd{Array 10 (Stored
Sint32)}.  Indexing into arrays is accomplished through the use of the \cd{(!)}
operator, and a valid index.

An index into an array has the type \cd{Ix}, which is parameterized by the size
of the array that it is indexing into.  The \cd{Ix n} type will only hold
values between zero and \cd{n}, which allows us to avoid run-time array bounds
checks, as in \ref{memareas}.

Additionally, we define a combinator \cd{arrayMap}, 

\lee{Trevor: Introduce the use of type-level nats here, and how they are used to
  enforce read/write safety for arrays. I'd discuss \cd{arrayMap} and \cd{Ix}
  here, too. (Maybe discuss possible extensions with decidable type-level
  arithmetic). Maybe mention \cd{CArray}s, too.}

\subsection{Procedures}
\lee{Trevor: describe the use of type-level lists to define procedures. Describe
the use of a type-class to ensure procedure calls in Ivory are type-correct.}

\subsection{Bit-Data}
\lee{James: talk about using TH to define bitdata.}

\subsection{Strings}
\lee{James}

\subsection{Maybe Types}
\lee{James: not sure, but maybe this would be useful to have?}

\subsection{Module System}
\lee{Lee: describe the module system, warts and all.}

As seen in Section~\ref{sec:ivory-overview}, Ivory's module system packages up
the collection of procedures, data declarations, etc. to be passed to a
back-end, such as the C code generator. The module system is implemented simply
as a writer monad that produces a list of abstract syntax values that are
processed by the various back-ends.

While the module system implementation is intuitive and simple, it is also one
of the greatest weaknesses of Ivory as an EDSL. \lee{XXX finish}

Values included in a module must be monomorphic; a type error results
otherwise. 

Because of our design decision 

\lee{make sure Pat's section show the Ivory module system.}






