\section{Ivory Semantics}
\label{sec:semantics}

In this section we discuss the more formal aspects of the Ivory
language, in particular, its semantics, and how this semantics is
modelled inside the Isabelle/HOL theorem prover\sjw{cite}.  We also
discuss the proof of type safety.  This model is available under the
\texttt{ivory-formal-model} directory in the ivory repository.

\begin{figure}[t]

\sjw{TODO}

\label{fig:syntax}
\caption{Concrete syntax of core Ivory \sjw{core?}}
\end{figure}

\sjw{more on why we want a model/motivations (?)}

\subsection{Modeling Ivory}

The formal model of Ivory concerns only the core language --- aspects
of the embedding into Haskell are not treated.  The basis for this
model is a typical typed imperative language with function calls,
references, and memory allocation (but not memory freeing).  

The heap is modeled as a list of \emph{regions}, each region a finite
map from \emph{offsets} to \emph{heap values}; a \emph{reference} then
contains both a \emph{region index} into the list of regions, and an
offset with the region.  

Operationally, the region list is extended on a function call, a
region being added to the end of the list, and shrunk on function
return, removing the last region.  Allocating an object extends the
current (last) region.

\sjw{don't really care?  maybe move to discussion}
Variables are immutable, in keeping with the Haskell implementation,
and thus variables are bound using a let-style statement.

\sjw{stratification of values into ref/non-ref makes things easy}

\subsection{Typing Ivory}

\begin{figure*}[t]

\sjw{TODO}

\label{fig:rules}
\caption{Selected semantic and typing rules.}
\end{figure*}

\begin{itemize}
\item Regions everywhere
\item type (region) schemas
\item Ignore must-return annotation?
\item Extra check on return type (frees in ret are in args) \sjw{is
    this actually needed?}
\item type substitutions (only on call?)
\end{itemize}

\subsection{Type Safety}

\begin{itemize}
\item Progress proof: pretty boring
\item Preservation: necessary/interesting invariants
\item other properties (obvious closure lemma/invariant?)
\item G, p |- e : t ==> frees t \in frees G
\item tsubst preserves well-formedness
\end{itemize}

\subsection{Nested Regions}

\begin{itemize}
\item Simpler syntax (no stratification)
\item Much more complex proofs/invariants
\item What are the main ideas (downward closed heap)?
\end{itemize}

\subsection{Discussion}

\begin{itemize}
\item modeling gap
\item various bugs in Ivory but not in Isabelle model
\item time estimate (approx 1 month?)
\item future work: first-class regions?
\item what have we gained?
\end{itemize}











% \lee{Simon: this is yours. Give the typing rules for Ivory, sketch the proofs of
%   progress and preservation, and describe this embedding in Isabelle. Also
%   describe the (work-in-progress) to extend Ivory with memory regions and
%   relaxed references. I'm expecting this'll be about 1-2 pages. Talk about a
%   little additional type-checking at value level (return statements), discovered
% from formal modeling.}
