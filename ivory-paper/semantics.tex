\newcommand{\coreivory}{Core Ivory}

\section{Ivory Semantics}
\label{sec:semantics}

In this section we discuss the more formal aspects of the Ivory
language, including its static and dynamic semantics.  We modeled a
simplified version of the Ivory language inside
Isabelle/HOL\sjw{cite}, henceforth \emph{\coreivory{}}; this
development is available under the \texttt{ivory-formal-model}
directory in the Ivory repository.  We present in this section a
semantics based upon the Isabelle/HOL development.
\lee{I'd remove where it's located or just mention that in the intro, with
  mentioning where the lang is.}

\lee{In the motivation, I think we should pose the question: ``in the previous
  section, we described our embedding of Ivory into the GHC type system and make
a claim that this guarantees memory safety. But the argument is complicated and informal.''}
Developing this model provides a number of benefits for a modest
investment --- we developed the model in under a person month, albeit
one of the authors has significant experience with Isabelle.  In
addition to the basic benefits formalisation provides, we can
experiment with extensions to Ivory.

In one such experiment, we extended the model to allow references in
the heap, a feature we avoided in the development of Ivory due to
soundness concerns.  While a simple extension to the syntax and
semantics of Ivory, the effort involved in extending the soundness
proofs was almost as much as developing the initial model.  

\newcommand{\sep}{\ |\ }

\newcommand{\syntaxtitle}[1]{\multicolumn{3}{l}{\textit{#1}}}
\begin{figure}[t]

\[
\begin{array}{crl}
\syntaxtitle{pure expressions}\\
e & ::= & 0 \sep{} 1 \sep{} \ldots{} \sep{} \texttt{true} \sep{} \texttt{false} \sep{} () \sep{} x \sep{} 
          e_1 \mathbin{\mathit{op}} e_2 \\
\syntaxtitle{impure expressions}\\
i & ::= & \texttt{pure}(e) \sep{} \texttt{alloc}(e) \sep{}
          \texttt{read}(e) \sep{} \texttt{write}(e_1, e_2)\\
\syntaxtitle{statements}\\
s & ::= & \texttt{skip} \sep{} \texttt{return}(e) \sep{} s_1; s_2 \\ 
  & |   & \texttt{if}(e)\texttt{\;then\;} s_1 \texttt{\;else\;} s_2 \\
  & |   & \texttt{for}(x = e_1; e_2; e_3) \{ s \} \\
  & |   & \texttt{let\;} x = i \texttt{\;in\;} s \\
  & |   & \texttt{let\;} x = f(e_1, \ldots{}, e_n) \texttt{\;in\;} s \\
\syntaxtitle{procedure definitions}\\
P & ::= & \texttt{proc\;} f(x_1, \ldots{}, x_n) \{ s \}
\\
\syntaxtitle{values}\\
v & ::= & 0 \sep{} 1 \sep{} \ldots{} \sep{} \texttt{true} \sep{} \texttt{false} \sep{} () \\
w & ::= & \texttt{stored}(v) \sep{} \texttt{ref}(r, n) \\
\syntaxtitle{environments}\\
E & \in{} & x \to w \\
\syntaxtitle{regions and heaps}\\
R & \in{} & \mathbb{N} \to w\\
H & ::=   & H, R \sep{} \cdot{}\\
\syntaxtitle{stacks}\\
F & ::= & \texttt{rframe}(x, E, s) \sep{} \texttt{sframe}(E, s) \\
S & ::= & F, S \sep{} \cdot{} \\
\syntaxtitle{configurations}\\
C & \in & H \times S \times E \times s \sep{} \texttt{finished}(v) \\
\syntaxtitle{types}\\
\rho   & \in & \textit{region variables}\\
\alpha & ::= & \texttt{nat} \sep{} \texttt{bool} \sep{} \texttt{unit} \\
\tau   & ::= & \texttt{storedt}(\alpha) \sep{} \texttt{reft}(\rho, \alpha)
\end{array}
\]

\sjw{values (in or out?), $\in$ or $\subseteq$?}

\label{fig:syntax}
\caption{Concrete syntax of \coreivory{}}
\end{figure}

\sjw{more on why we want a model/motivations (?)}
\lee{i don't think we have to motivate the need for semantics with icfp!}

\subsection{Syntax}

The syntax for \coreivory{} is given in \autoref{fig:syntax}.
\coreivory{} is based upon a typical typed imperative language with
function calls, references, and memory allocation (but not memory
deallocation).  \coreivory{} attempts to stay faithful to Ivory
wherever possible, and so variables are let-bound with forms for
binding the result of expression evaluation and function
calls. Furthermore, \coreivory{} expressions are stratified into
\emph{pure} and \emph{impure}, the latter allowing operations on the
heap: allocation, reading, and writing references.

Ivory uses regions to manage memory.  Thus, the heap is modeled as a
list of \emph{regions}, each region a finite map from \emph{offsets},
modeled as natural numbers, to \emph{stored values}; Ivory does not
allow references in heap allocated values, and so a stored value is
any value which is not a reference. A \emph{reference} contains both a
\emph{region index} into the list of regions, and an offset with the
region.  To simplify the presentation, we will use $H(r, n)$ to denote
the value at offset $n$ in the $r$th region of $H$, and similarly with
updates.

As with values, types classifying values in \coreivory{} are
stratified into storable and reference types; a reference type
$\texttt{reft}(\rho, \alpha)$ is a reference to an object of type
$\alpha$ in region $\rho$, where $\alpha$ is not a reference.

\sjw{talk about structs and arrays?}

\subsection{Operational Semantics}


\newcommand{\stepsX}[2]{\models #1 \longmapsto{} #2}
\newcommand{\stepsXX}[3]{\models #2 \longmapsto^{#1} #3}
\newcommand{\steps}[4]{\stepsX{#1; #2}{#3; #4}}

\newcommand{\denoteexp}[2]{\llbracket{}#1\rrbracket{}#2}

\newcommand{\stepsH}[2]{#1 \longmapsto_I #2}
\newcommand{\hsteps}[5]{#1 \models \stepsH{#2; #3}{#4; #5}}

\newcommand{\wfstmt}[5]{#1; #2; #3 \vdash_s #4 : #5}
\newcommand{\wfexp}[3]{#1 \vdash_e #2 : #3}
\newcommand{\wfimp}[4]{#1; #2 \vdash_i #3 : #4}

\newcommand{\rulestitle}[1]{\textnormal{#1}\hspace*{\fill}}

\begin{figure*}[ht]

\begin{mathpar}

%% Pure expressions

% \sjw{all but var lookup return stored(x)}
\rulestitle{Semantics}\\
\denoteexp{e}{E} = \ldots

\and
%% Impure expression

\inferrule{\denoteexp{e}{E} = w}{\hsteps{E}{H}{\texttt{pure}(e)}{H}{w}}
\and
\inferrule{\denoteexp{e}{E} = \texttt{stored}(v)}{\hsteps{E}{H, R}{\texttt{alloc}(e)}{H, R[p \mapsto{} v]}{\texttt{ref}(|H|, p)}}(p \notin{} \mathrm{dom}(R))
\and
\inferrule{\denoteexp{e}{E} = \texttt{ref}(r, n)}{ \hsteps{E}{H}{ \texttt{read}(e) }{H}{\texttt{stored}(H(r, n))} }
\and
\inferrule{\denoteexp{e_1}{E} = \texttt{ref}(r, n) \\ \denoteexp{e_2}{E} = \texttt{stored}(v)}{ \hsteps{E}{H}{ \texttt{write}(e_1, e_2) }{H[(r, n) \mapsto v]}{\texttt{stored}(())} }

\\
%% Statements

\inferrule{ {} }{\steps{(H; \texttt{sframe}(E, s), S; \_)}{\texttt{skip}}{(H; S; E)}{s}}
\and
\inferrule{ {} }{\steps{(H; \texttt{sframe}(\_, \_), S; E)}{\texttt{return}(e)}{(H; S; E)}{\texttt{return}(e)}}
\and
\inferrule{ \denoteexp{e}{E_0} = w }{\steps{(H, \_; \texttt{rframe}(x, E, s), S; E_0)}{\texttt{return}(e)}{(H; S; E[x \mapsto{} w])}{s}}
\and
\inferrule{ \denoteexp{e}{E} = w }{ \stepsX{(\_; \cdot; E); \texttt{return}(e)}{\texttt{finished}(w)} }
\and
\inferrule{ {} }{\steps{(H; S; E)}{ (s_1; s_2) }{(H; \texttt{sframe}(E, s_2), S; E)}{s_1}}
\and

\inferrule{ \denoteexp{e}{E} = \texttt{stored}(\texttt{true}) }{\steps{(H; S; E)}{ \texttt{if}(e)\texttt{\;then\;} s_1 \texttt{\;else\;} s_2 }{(H; S; E)}{s_1}}
\and
\inferrule{ \denoteexp{e}{E} = \texttt{stored}(\texttt{false}) }{\steps{(H; S; E)}{ \texttt{if}(e)\texttt{\;then\;} s_1 \texttt{\;else\;} s_2 }{(H; S; E)}{s_2}}
\and
\inferrule{ \denoteexp{e_1}{E} = w }{\steps{(H; S; E)}{ \texttt{for}(x = e_1; e_2; e_3) \{ s \} }{(H, S, E[x \mapsto{} w])}%
{\texttt{if}(e_2)\texttt{\;then\;} (s; \texttt{for}(x = e_3; e_2; e_3) \{ s \}) \texttt{\;else\;} \texttt{skip} }}
\and
\inferrule{ \hsteps{E}{H}{i}{H'}{w} }{\steps{(H; S; E)}{\texttt{let\;} x = i \texttt{\;in\;} s}{(H'; S; E[x \mapsto{} w])}{s}}
\and
\inferrule{ \denoteexp{e_i}{E} = w_i \\ \texttt{proc\;} f(x_1, \ldots, x_n)\{ \mathit{body} \} \in \texttt{Procs} }{\steps{(H; S; E)}{\texttt{let\;} x = f(e_1, \ldots{}, e_n) \texttt{\;in\;} s}{(H, \emptyset; \texttt{rframe}(x, E, s), S; [x_1 \mapsto{} w_1. \ldots{}, x_n \mapsto{} w_n])}{\mathit{body}}}

\\
\rulestitle{Typing rules}\\

% wfimp
\inferrule{ \wfexp{\Gamma}{e}{\tau} }{ \wfimp{\Gamma}{\rho}{\texttt{pure}(e)}{\tau} }b
\and
\inferrule{ \wfexp{\Gamma}{e}{\alpha} }{ \wfimp{\Gamma}{\rho}{\texttt{alloc}(e)}{\texttt{reft}(\rho, \alpha)} }
\and
\inferrule{ \wfexp{\Gamma}{e}{\texttt{reft}(\rho, \alpha)} }{ \wfimp{\Gamma}{\rho}{\texttt{read}(e)}{\texttt{storedt}(\alpha)} }
\and
\inferrule{ \wfexp{\Gamma}{e_1}{\texttt{reft}(\rho, \alpha)} \\ \wfexp{\Gamma}{e_2}{\alpha} }{ \wfimp{\Gamma}{\rho}{\texttt{write}(e_1, e_2)}{\texttt{storedt}(\texttt{unit})} }

\\
%% wfstmt
\inferrule{ }{ \wfstmt{\Gamma}{\Psi}{\rho}{\texttt{skip}}{\tau} }
\and
\inferrule{ \wfexp{\Gamma}{e}{\tau}  }{ \wfstmt{\Gamma}{\Psi}{\rho}{\texttt{return}(e)}{\tau} }
\and
\inferrule{ \wfstmt{\Gamma}{\Psi}{\rho}{s_1}{\tau} \\ \wfstmt{\Gamma}{\Psi}{\rho}{s_2}{\tau} }{ \wfstmt{\Gamma}{\Psi}{\rho}{s_1; s_2}{\tau} }
\and
\inferrule{ \wfexp{\Gamma}{e}{\texttt{bool}} \\ \wfstmt{\Gamma}{\Psi}{\rho}{s_1}{\tau} \\ \wfstmt{\Gamma}{\Psi}{\rho}{s_2}{\tau} }%
{ \wfstmt{\Gamma}{\Psi}{\rho}{\texttt{if}(e)\texttt{\;then\;} s_1 \texttt{\;else\;} s_2 }{\tau} }
\and
\inferrule{ \wfexp{\Gamma}{e_1}{\sigma} \\ \wfexp{\Gamma[x \mapsto \sigma]}{e_2}{\texttt{bool}} \\ \wfexp{\Gamma[x \mapsto \sigma]}{e_3}{\sigma}
\\ \wfstmt{\Gamma[x \mapsto \sigma]}{\Psi}{\rho}{s}{\tau}}%
{ \wfstmt{\Gamma}{\Psi}{\rho}{ \texttt{for}(x = e_1; e_2; e_3) \{ s \} }{\tau} }
\and
\inferrule{ \wfimp{\Gamma}{\rho}{i}{\sigma} \\  \wfstmt{\Gamma[x \mapsto \sigma]}{\Psi}{\rho}{s}{\tau} }%
{ \wfstmt{\Gamma}{\Psi}{\rho}{\texttt{let\;} x = i \texttt{\;in\;} s}{\tau} }
\and
\inferrule{ \wfexp{\Gamma}{e_i}{\sigma_i} \\
            \Psi(f) = \texttt{fun}(\sigma, (\sigma_1, \ldots, \sigma_n))\\
            \wfstmt{\Gamma[x \mapsto \sigma]}{\Psi}{\rho}{s}{\tau} }%
{ \wfstmt{\Gamma}{\Psi}{\rho}{\texttt{let\;} x = f(e_1, \ldots{}, e_n) \texttt{\;in\;} s}{\tau} }

\\
\inferrule{ \forall{}f \in \textrm{dom}(\Psi), \\\\ \Psi(f) = \texttt{fun}(\tau, (\sigma_1, \ldots, \sigma_n))\\      
           \rho \textrm{ fresh} \\     
            \texttt{proc\;} f(x_1, \ldots, x_n)\{ \mathit{s} \} \in \texttt{Procs} \\            
            \texttt{frees}(\tau) \subseteq \texttt{frees}(\sigma_1) \cup \ldots \cup \texttt{frees}(\sigma_n) \\
            \wfstmt{[x_1 \mapsto \sigma_1, \ldots, x_n \mapsto \sigma_n]}{\Psi}{\rho}{s}{\tau} }%
{ \vdash \texttt{Procs} : \Psi }

\end{mathpar}

\label{fig:rules}
\caption{Selected semantic and typing rules. The semantics and typing rules for expressions ($\denoteexp{e}{E}$ and $\wfexp{\Gamma}{e}{\tau}$) are standard and are so omitted.}
\end{figure*}

\sjw{mention missing rules in the figure} \coreivory{}'s semantics,
presented in \autoref{fig:rules}, is modeled as an abstract
machine\sjw{cite?} over configurations.  The judgement
\[
\stepsX{C}{C'}
\]
states that configuration $C$ transitions to configuration $C'$ and
relies on the denotation of expressions (not shown) $\denoteexp{e}{E}$
and the semantics of impure expressions
\[
\hsteps{E}{H}{i}{H'}{w}
\]
stating that execution of the impure expression $i$ in environment $E$
and heap $H$ yields a new heap $H'$ and value $w$.

A configuration consists of a heap, a stack, an environment, and a
statement. The stack contains continuations for both function calls
and statement sequences; the environment maps variables to values. The
semantics of sequencing is slightly non-standard as variables are
let-bound rather than assigned, and so statement sequencing preserves
the environment across execution of the first statement.

Operationally, the heap is extended on a function call, an empty
region being added to the end of the list, and shrunk on function
return, removing the last region.  Allocating an object extends the
current (last) region.

A configuration is stuck if there is no available transition.  For
instance, an attempted heap access or update where the region index
does not exist or at an offset which has not been allocated will
result in a stuck configuration.  In particular, accessing a region
after it has been removed will result in a stuck state.  

\sjw{make stuc states more explicit?}
% \sjw{stratification of values into ref/non-ref makes things easy}

\subsection{Typing Ivory}

The typing rules for \coreivory{} statements are given in
\autoref{fig:rules}.  The typing judgement 
\[
\wfstmt{\Gamma}{\Psi}{\rho}{s}{\tau}
\]
holds when the statement $s$ is well-formed under the variable
environment $\Gamma$, procedure environment $\Psi$, current region
$\rho$, with any return statements returning values of type $\tau$.

The region variable $\rho$ represents the current region and is used
when checking memory allocation.  In particular, we may derive the
following typing rule
\[
\inferrule{ 
\wfexp{\Gamma}{e}{\alpha} \\
\wfstmt{\Gamma[x \mapsto \texttt{reft}(\rho, \alpha)]}{\Psi}{\rho}{s}{\tau} }%
{ \wfstmt{\Gamma}{\Psi}{\rho}{\texttt{let\;} x = \texttt{alloc}(e) \texttt{\;in\;} s}{\tau} }
\]
where the body of the let statement is checked under the additional
assumption that the variable $x$ has reference type
$\texttt{reft}(\rho, \alpha)$.  

The typing rule for procedure bodies ensures that this region variable
is fresh; this constraint, together with the constraint that region
variables in the procedure's return type must occur in an argument
type, ensures that references cannot escape the scope in which they
were allocated.  These constraints are then fundamental to the type
safety of \coreivory{} programs.

\subsection{Type Safety}

We prove type safety, that is, well-typed programs do not get stuck,
by proving the usual progress and preservation lemmas.  As is
common\sjw{cite?}  with type safety for imperative languages, we
define auxiliary well-formedness invariants on configurations.  The
progress lemma then states that well-formed configurations are not
stuck, and preservation states that well-formed configurations
transition to well-formed configurations.

\sjw{name this?}
Formally, we prove, assuming 
\[
\begin{array}{l}
\vdash \texttt{Procs} : \Psi \\
\Psi(\texttt{main}) = \texttt{fun}(\texttt{nat}, ())
\end{array}
\]
that there exists some number of steps $n$ and value $v \in \mathbb{N}$ such that
\[
\stepsXX{n}{(\cdot; \cdot; \cdot); \texttt{let\;} x = \texttt{main}() \texttt{\;in\;} \texttt{return}(x)}{\texttt{finished}(v)}
\]
or, for all $n$ there is a well-formed configuration $C$ such that
\[
\stepsXX{n}{(\cdot; \cdot; \cdot); \texttt{let\;} x = \texttt{main}() \texttt{\;in\;} \texttt{return}(x)}{C}
\]
Informally, a well-formed program, when called via the \texttt{main}
procedure, will either terminate in a finite number of steps, or will
diverge through well-formed configurations.

The well-formedness invariants are typical, and are based upon a
well-formed value judgement.  For our purposes, the interesting rule
here is for references
\[
\inferrule{\Delta(\rho) = r \\ \Theta(r, n) = \alpha }%
{\Theta; \Delta \vdash \texttt{ref}(r, n) : \texttt{reft}(\rho, \alpha)}
\]
which links reference types to reference values through the region
environment $\Delta$, mapping region variables to region indices, and
heap type $\Theta$, mapping region indices and offsets to types.
Well-formed heaps and environments then follow point-wise from this
judgement.  Well-formedness of stacks follows, ensuring that each
continuation on the stack is well-formed.

A well-formed configuration, in addition to the well-formedness of the
heap, stack, variable environment, and current statement, constrains
the region environment ($\Delta$).  In particular, the variable
representing the current region must be mapped to the length of the
current heap, every region index in the range of $\Delta$ must be
below this length, and the type variables occurring in the variable
environment $\Gamma$ must be mapped by $\Delta$.

The progress lemma follows from well-formedness.  The preservation
proof, as usual, is the trickier of the two proofs.  In particular,
the case for return involves showing that the various configuration
members are well-formed under a heap where the last region has been
removed from the heap.  This involves showing that references are
well-formed under this smaller heap which follows from the stack and
region environment well-formedness invariants.  The case for function
calls is also involved, although this is primarily due to the
instantiation of the type variables in the type of the called
function.

\subsection{Discussion}

The type safety proofs are greatly simplified by the restriction of
heap values to non-references: it is trivially true, for example, that
the heap is well-formed after a return as the well-formedness of
non-reference values does not depend on the type of other heap
elements.

We extended the Isabelle model to remove the stratification of values,
allowing references to appear in the heap.  Syntactically, this
simplifies the language, at the expense of more complicated proofs.
For example, we extend the well-formedness of heaps to require that
heaps are downward closed with respect to region references: that is,
references in the heap refer only to regions up to and including the
containing region.  This work extended the proof development from
approximately 2500 lines of proof to 3300 lines of proof, and took
approximately 2 person weeks to finish.

\sjw{move to end?}
Develop0ing this model uncovered a bug in the Ivory embedding into
Haskell, namely that the Haskell erroneously allowed functions to
end without encountering a return statement.  Thus, a program could
claim to return, say, a valid index but in reality return an
out-of-bounds index, breaking memory safety.  In the model these
check is part of the type checking rules (although elided for clarity
in this paper) but is implemented as a post-processing check in the
Haskell implementation.  

The existence of the formal model is not a guarantee that the Ivory
implementation is sound, however: there is no formal link between the
implementation and the model.  For example, we discovered a number of
bugs due to the way in which references are initialized: these bugs
are impossible in the model, but allowed in the implementation.  

In addition, \coreivory{} covers a subset of the Ivory language,
currently missing data structures and arrays.  We are working to
reduce this gap; furthermore, in future, we plan to investigate making
the core of the implementation more closely resemble \coreivory{}.
Doing so would have the additional benefit of allowing us to verify
the correctness of operations performed over the core AST.

We plan on investigating first-class regions as a further extension to
the model.  This feature would allow programmers to name and pass
around allocation contexts, allowing helper functions to, for example,
allocate and return objects in a parent functions region.  While we
are confident that this extension is sound, \emph{proving} it sound
allows a much greater confidence in the implementation: memory
deallocation is notoriously easy to get wrong, so having a formal
proof of soundness would be greatly comforting. \sjw{comforting?}


% \lee{Simon: this is yours. Give the typing rules for Ivory, sketch the proofs of
%   progress and preservation, and describe this embedding in Isabelle. Also
%   describe the (work-in-progress) to extend Ivory with memory regions and
%   relaxed references. I'm expecting this'll be about 1-2 pages. Talk about a
%   little additional type-checking at value level (return statements), discovered
% from formal modeling.}
