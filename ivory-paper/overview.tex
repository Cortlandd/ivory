\section{Ivory Overview}
\label{sec:ivory-overview}

\lee{need lead-in here. What does Ivory do? What's the motivation for it's
  design (I like to motivate it with the JPL's Power of 10 rules).}


\subsection{Ivory Statements}

\lee{I'd suggest we elide types in this section, to the extent possible.}

Ivory has a simple statement language embedded in a monad. Unlike in C, all
reference reading and writing must explicitly take place in a statement, which
eliminates large classes of unintuitive and undefined behaviors.

Ivory expressions are immutable. References (non-nullable pointers) may be
read and written using the \cd{deref} and \cd{store} statements, respectively.
The following Haskell function takes a reference to a signed 32 bit integer value,
and returns a statement which increments the value of the reference.

\begin{code}
incr_ref :: Ref s (Stored Sint32) -> Ivory eff ()
incr_ref r = do
    v <- deref r
    store r (v + 1)
\end{code}

Ivory users can create global memory areas, like in C:
\begin{code}
ex_global_area :: MemArea (Stored Sint32)
ex_global_area = area "ex_global" (Just (ival 42))

ex_global :: Ref Global (Stored Sint32)
ex_global = addrOf ex_global_area
\end{code}

There is a separate concept of the memory area \cd{:: MemArea a}, which
exists as part of a module def (as we'll explain below), and a reference
to that area, which is a value that can be mutated.

In addition to global memory areas, we can create memory areas on the stack.
The \cd{local} primitive allocates a value on the stack, and returns a reference to
that area.

To use allocation in an Ivory statement, the \cd{eff} parameter must be
constrained to have a given allocation scope. The following function creates an integer
on the stack with an initial value of 3, increments it, and checks that the resulting
value is equal to 4.

\begin{code}

uses_allocation :: (GetAlloc eff \mytilde Scope s)
                => Ivory eff IBool
uses_allocation = do
  r <- local (ival (3 :: Sint32))
  incr_ref r
  v <- deref r
  return (v ==? 4)

\end{code}


\begin{code}
ex_incr_proc :: Def ('[Sint32]:->IBool)
ex_incr_proc = proc "ex_incr" $ \a -> body $ do
  incr_ref ex_global
  v <- deref ex_global
  ret (v >? a)
\end{code}

\lee{You need to put the English description before the program.}

We can wrap up statements into a procedure. Procedures can take arguments
passed in as a lambda, and optionally return an ivory value with the \cd{ret}
statement.

\begin{code}
ex_module :: Module
ex_module = package "example" $ do
  incl ex_incr_proc
  private $ do
    defMemArea ex_global_area
\end{code}

and then collect procedures and memory areas into a module, which is then
compiled to two files, \cd{example.h} containing the public interface and
\cd{example.c} containing the implementation.

\subsection{Data structures}

Ivory provides a collection of atomic (\cd{Stored}) types which may be composed
as product types. Arrays are specified by the type \cd{Array n a}, where \cd{n}
is a type natural specifying the array length. Because length is specified in
the type system, all arrays must have a known and finite length. Array
members are accessed through the \cd{!} operator, which enforces that array
offsets are specified by a special type \cd{Ix n}, which only permits expressing
natural numbers in \cd{0 <= Ix n < n}.

\begin{code}
array_ex :: Ref s (Array 10 (Stored Sint32))
         -> Ivory eff ()
array_ex a = incr_ref (a ! 5)
\end{code}

Traditional C structures are defined using a quasiquoter to specify the field
names and their types. Each field name can be used with the \cd{\mytilde>} operator
which takes a reference to a struct and a field name in that struct, and gives
a ref of the field's type. The following code declares a structure type named
\cd{position}, allocates a position on the stack with an initial value, and then
shows some basic operations on elements in the structure.

\begin{code}
[ivory|
struct position
  { latitude  :: Stored IFloat
  ; longitude :: Stored IFloat
  ; altitude  :: Stored Sint32
  }
|]

struct_ex :: (GetAlloc eff \mytilde Scope s) => Ivory eff ()
struct_ex = do
  s <- local (istruct [ latitude .= ival 45.52
                      , longitude .= ival (-122.68)
                      , altitude .= ival 1524 ])
  lat <- deref (s \mytilde> latitude)
  lon <- deref (s \mytilde> longitude)
  incr (s \mytilde> altitude)
\end{code}


\subsection{Control structures}

Ivory has a few simple control flow primitives. For boolean conditions,
the \cd{ifte\_} primitive is a statement that takes an \cd{IBool} and two
branches of type \cd{Ivory eff ()}. For conditions that do not cause side
effects, the \cd{?} operator gives an expression from an IBool and a tuple
of two expressions.

\begin{code}
abs :: Def('[Sint16] :-> Sint16)
abs = proc "abs" $ \v -> body $ do
  ifte_ (v <? 0)
    (ret (-1*v))
    (ret v)

abs2 :: Def('[Sint16] :-> Sint16)
abs2 = proc "abs2" $ \v -> body $ do
  ret $ (v <? 0) ? ((-1*v), v)
\end{code}

Ivory has two types of loops: unbounded loops, using the \cd{forever} primitive,
are for control structures that should never terminate, such as OS tasks. All
other loops must specify an upper bound at the type level, using the \cd{Ix n} type
we used for array indexing. The \cd{arrayMap} loop is run once for every value
in the range given by \cd{Ix n}. The following loop computes the sum of the
numbers from 1 to 16:

\begin{code}
loop_ex :: (GetAlloc eff \mytilde Scope s) => Ivory eff Sint32
loop_ex = do
  a <- local (iarray [1..])
  v <- local (ival 0)
  arrayMap $ \(ix :: Ix 16) -> do
    aix <- deref (a ! ix)
    vv <- deref v
    store v (vv + aix)
  deref v
\end{code}

Loops may be terminated early using the \cd{break} statement.

\begin{code}
predicates_ex :: Def('[ IFloat ] :-> IFloat)
predicates_ex = proc "predicates_ex" $
    \i -> requires (i >? 0)
        $ ensures (\r -> r >? 0)
        $ body
        $ do (assert (i /=? 0))
             ret (i + 1))
\end{code}

Procedures may also specify preconditions and postconditions, and users may
specify assertions as statements. The Ivory compiler can emit run-time
assertions to enforce these conditions, or use a model checker backend to
statically verify they are satisified.

\begin{code}

print_ex_module :: Module
print_ex_module = package "print_ex" $ do
  incl printf_none
  incl printf_sint32
  incl print_proc

printf_none :: Def('[IString] :-> Sint32)
printf_none  = importProc "printf" "stdio.h"

printf_sint32 :: Def('[IString, Sint32] :-> Sint32)
printf_sint32  = importProc "printf" "stdio.h"

print_proc = Def('[]:->())
print_proc = proc "print_proc" $ body $ do
  _ <- call printf_none "hello, world!\n"
  _ <- call printf_sint32 "print an integer: \%d" 42
  return ()
\end{code}

Ivory can interact with externally defined C functions and global
variables. The \cd{importProc} primitive allows the user to declare an external
procedure, and ensures the correct header file is included by the generated
code.

Ivory can only import and use functions that have a valid Ivory type signature.
Some polymorphic C functions may have multiple valid Ivory types.

\subsection{Toolchain Use}
The Ivory compiler is a haskell function that takes a list of \cd{Module}s,
parses command line options, and writes generated C source and header files to
a directory given by those options.

The compiler's second argument is a list of \cd{Artifact}s. Artifacts are a
datatype for an arbitrary haskell string and a filename, indicating the contents
of a non-Ivory-generated file to be written to the output directory. In
practice, this is used to write Makefiles, linker scripts, and debug output
specified by the user.

There are also related functions exposed to the user that allow the parsing
of command line options to be separated from the compile step, where desired.

\begin{code}
import Ivory.Compile.C (compile)

main :: IO ()
main = compile [ ex_module, print_ex_module ] []
\end{code}






