\section{Ivory Overview}
\label{sec:ivory-overview}

\lee{Here, we'll give the main overview of Ivory without worrying about
  describin how it is implemented. We want to say what you can and cannot ``do''
  in Ivory. Also, let's give a taste of the programmer interface, e.g., how to
  define a struct, a procedure, bit data, write an array map, deal with extern
  data/procedures. I expect this'll be about 1 page, maybe 2. In some sense, the
content will be kinda like Iavor's intro in his Habbit/Hobbit/whatever
paper. It'll be more of an informal, almost tutorial style.}

\lee{We should cover everything describe in
  Section~\ref{sec:ivory-embedding}. In addition, we should cover pre-post
  condition, assertions, bit-data, safe and casting.}

\lee{I think a good guide for this section is Sec. 2 from the Shake paper:
  \url{http://community.haskell.org/~ndm/downloads/paper-shake_before_building-10_sep_2012.pdf}}

\lee{Pat: maybe you could take a stab at this one?}


\subsection{Ivory Statements}

Ivory has a simple statement language embedded in a monad. Unlike in C, all
reference reading and writing must explicitly take place in a statement, which
eliminates large classes of unintuitive and undefined behaviors 

\begin{code}
incr_ref :: Ref s (Stored Sint32) -> Ivory eff ()
incr_ref r = do
    v <- deref r
    store r (v + 1)
\end{code}

In addition to global memory areas, we can create memory areas on the stack.
The first parameter to \cd{Ref} specifies the allocation scope, which is either
\cd{Global} or \cd{Stack s}. The \cd{local} primitive permits allocating a value
locally (on the stack associated with the parent procedure), and tags the
\cd{Ref} with a type that prevents a pointer to the stack to escape the parent
procedure scope, e.g. by being copied to a global memory area.

To use allocation in an Ivory statement, the \cd{eff} parameter must be
constrained to have a given allocation scope:

\begin{code}
local :: (IvoryArea a, GetAlloc eff ~ Scope s)
      => Init a -> Ivory eff (Ref (Stack s) a)

uses_allocation :: (GetAlloc eff ~ Scope s) => Ivory eff IBool
uses_allocation = do
  r <- local (ival (3 :: Uint8))
  incr_ref r
  v <- deref r
  return (v ==? 4)

\end{code}

Ivory users can create global memory areas, just like in c:
\begin{code}
ex_global_area :: MemArea (Stored Sint32)
ex_global_area = area "ex_global" (Just (ival 42))

ex_global :: Ref Global (Stored Sint32)
ex_global = addrOf ex_global_area
\end{code}

There is a separate concept of the memory area \cd{:: MemArea a}, which
exists as part of a module def (as we'll explain below), and a reference
to that area, which is a value that can be 

\begin{code}
ex_incr_proc :: Def ('[Sint32]:->IBool)
ex_incr_proc = proc "ex_incr" $ \a -> body $ do
  incr_ref ex_global
  v <- deref ex_global
  ret (v >? a)
\end{code}

we can wrap up statements into a procedure. Procedures can take arguments,
as specified by the type level list, and return either a non-ref value or
\cd{()}, which works just like C's \cd{void}. Arguments can be \cd{Ref}s, but
the type system ensures the scope parameter of those refs is not the same type
as refs created inside the \cd{body}.

\begin{code}
ex_module :: Module
ex_module = package "example" $ do
  incl ex_incr_proc
  private $ do
    defMemArea ex_global_area
\end{code}

and then collect procedures and memory areas into a module, which is then
compiled to two files, \cd{example.h} containing the public interface and
\cd{example.c} containing the implementation.

\subsection{Data structures}

Ivory provides a collection of atomic (\cd{Stored}) types which may be composed
as product types. Arrays are specified by the type \cd{Array n a}, where \cd{n}
is a type natural specifying the array length. Because length is specified in
the type system, all arrays must have a known and finite length. Array
members are accessed through the \cd{!} operator, which enforces that array
offsets are specified by a special type \cd{Ix n}, which only permits expressing
natural numbers in \cd{0 <= Ix n < n}.

\begin{code}
array_ex :: (GetAlloc eff ~ Scope s) => Ivory eff ()
array_ex = do
  (a :: Ref (Stack s) (Array 10 (Stored Sint32))) <- local (iarray [0..])
  incr (a ! 5)
  -- incr (a ! 10) -- Impossible to write, error caught at compile time
  v <- deref (a ! 5)
  assert (v ==? 6)

\end{code}

Traditional C structures are defined using a quasiquoter to specify the field
names and their types. Each field name can be used with the \cd{~>} operator
which takes a reference to a struct and a field name in that struct, and gives
a ref of the field's type.

\begin{code}
[ivory|
struct position
  { latitude  :: Stored IFloat
  ; longitude :: Stored IFloat
  ; altitude  :: Stored Sint32
  }
|]

struct_ex :: (GetAlloc eff ~ Scope s) => Ivory eff ()
struct_ex = do
  s <- local (istruct [ latitude .= ival 45.52
                      , longitude .= ival (-122.68)
                      , altitude .= ival 1524 ])
  lat <- deref (s ~> latitude)
  lon <- deref (s ~> longitude)
  incr (s ~> altitude)
\end{code}


\subsection{Control structures}

Ivory has a few simple control flow primitives. For boolean conditions,
the \cd{ifte_} primitive is a statement that takes an \cd{IBool} and two
branches of type \{Ivory eff ()}. For conditions that do not cause side
effects, the \cd{?} operator gives an expression from an IBool and a tuple
of two expressions.

\begin{code}
abs :: Def('[Sint16] :-> Sint16)
abs = proc "abs" $ \v -> body $ do
  ifte_ (v <? 0)
    (ret (-1*v))
    (ret v)

abs2 :: Def('[Sint16] :-> Sint16)
abs2 = proc "abs2" $ \v -> body $ do
  ret $ (v <? 0) ? ((-1*v), v)
\end{code}

Ivory has two types of loops: unbounded loops, using the \cd{forever} primitive,
are for control structures that should never terminate, such as OS tasks. All
other loops must specify an upper bound at the type level, using the \cd{Ix n} type
we used for array indexing. The \cd{arrayMap} loop is run once for every value
in the range given by \cd{Ix n}. The following loop computes the sum of the
numbers from 1 to 16:

\begin{code}
loop_ex :: (GetAlloc eff ~ Scope s) => Ivory eff Sint32
loop_ex = do
  a <- local (iarray [1..])
  v <- local (ival 0)
  arrayMap $ \(ix :: Ix 16) -> do
    aix <- deref (a ! ix)
    vv <- deref v
    store v (vv + aix)
  deref v
\end{code}

Loops may be terminated early using the \cd{break} statement.

