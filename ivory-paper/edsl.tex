\section{A Safe-C EDSL}
\label{sec:edsl}

In this section, we discuss the following issues we encountered when
implementing an EDSL targeted toward constructing realistic embedded
systems, notably integrating a C-like concrete syntax into the EDSL,
error-reporting, and uses of macro programming specifically tailored
to embedded system programming.

\paragraph{Concrete Syntax}

A benefit of the EDSL approach is that it relieves the language developer from
having to define and implement front-end syntax. However, that also generally
means that only users of the host language will be attracted to using the EDSL.
We want C/C++ developers to use Ivory!

\begin{figure}[h!]
\begin{code}
void mapProc(*uint8_t[4] arr, uint8_t x) {
  map ix {
    let v = arr@ix;
    *v = *v + x;
  }
}
\end{code}
  \caption{Concrete Syntax for Ivory}
  \label{fig:concrete}
\end{figure}

\jamey{Might explain why an array type like ``arr'' needs an ``extra'' star in
its type. Could cite Dennis Ritchie's ``The Development of the C Language''
(\url{http://www.lysator.liu.se/c/chistory.ps}) to explain how C got the weird
array type semantics in the first place...}

We developed a concrete C-like syntax for Ivory, which facilitated use by
Boeing, mentioned in Section~\ref{sec:introduction}. We have
already seen specific uses of quasi-quotation in Ivory to define a concrete
syntax for structs (Section~\ref{sec:area}) and bit-data
(Section~\ref{sec:bitdata}). In addition, we implemented a quasi-quoter for the entire
Ivory language. An example quasi-quoted Ivory procedure is shown in
Figure~\ref{fig:concrete}. The procedure is the equivalent of the procedure with
the same name in Section~\ref{sec:control}.

A quasi-quoted Ivory program is guaranteed to be type-safe, since the generated
Haskell program is type-checked an important feature of the quasi-quoted
language is that it automatically generates the appropriate type signatures for
Ivory programs, relieving the programmer from doing so. The quasi-quoter also
generates Ivory modules automatically and guarantees that procedure names match
their Haskell identifiers, obviating the problems discussed in
Section~\ref{sec:modules}. The quasi-quoter supports anti-quotation, so that
Haskell can still be used as a macro language. All of Boeing's development in
Ivory is via the quasi-quoter.

Implementing a quasi-quoter means that we have defined a lexer and parser for
the concrete syntax. Because Ivory is deeply embedded in Haskell, there is a
concrete data type (i.e., the abstract syntax tree (AST)) over which
optimizations and back-ends are implemented. The distance from Ivory as an EDSL
and a stand-alone compiler is surprisingly small, essentially requiring a
type-checker and a front-end targeting the AST directly. In this manner, we are
able to ``grow'' a compiler, from an EDSL to a stand-alone system.


\paragraph{Haskell as Macro Language}
C programmers are accustomed to using the C Pre-Processor (CPP) as a limited
macro and metaprogramming system. CPP is significantly limited: it was
originally designed to prohibit recursion during macro expansion (although
C99's variadic macros have been abused to implement near-arbitrary
recursion~\cite{cpp-turing}). Widely used tools such as lex~\cite{lex} and
yacc~\cite{yacc} demonstrate other approaches to metaprogramming in C, and some
projects such as libxcb~\cite{xcb} have produced specialized metaprogramming
tools in languages such as m4~\cite{m4}.

Embedding Ivory within Haskell enables using the full power of Haskell as a
type-safe metaprogramming environment for C. We have used this power in a
variety of ways, specifically beneficial to embedded development.

\begin{itemize}
%% \item Our Tower component architecture allows compositional specification of
%%   components and their message-passing channels. When instantiating a particular
%%   configuration of a collection of components, the Tower backend performs
%%   whole-program analysis of the data-flow graph to statically allocate exactly
%%   as much buffer space as necessary, avoiding dynamic memory allocation.
\item In embedded systems, a peripheral device's clock may be derived from a
  platform-wide clock by a programmable clock divider. However, the platform
  clock may be configurable. Configuring, say, a serial port to operate at
  115200bps, may require a complex search over the space of clock divider and
  other configuration settings based on the current platform clock. But these
  settings will be the same on every run for a given application, so we can
  precompute them in Haskell.
\item We frequently use Haskell type-classes to provide function overloading for
  Ivory. For example, the ivory-serialize library provides a ``Packable''
  typeclass capturing the code needed to serialize or deserialize user-specified
  types. This style of overloading is resolved during Haskell run-time, so the
  generated C is monomorphized.
\item Mathematical metaprogramming has been a particularly powerful tool when combined
with Ivory. Edward Kmett's automatic differentiation~\cite{ad} and linear
algebra~\cite{linear} packages, which depend primarily on the Haskell Prelude's
numeric typeclasses, operate transparently over Ivory expressions---despite
having been designed independently. This allowed us to build a stand-alone
package, ``estimator''~\cite{estimator}, which automatically derives an Extended
Kalman Filter from a concise model written in pure Haskell. Like linear and ad,
the estimator package is independent of Ivory, but when evaluated over an Ivory
floating-point type, the result is a C implementation of the derived Kalman
filter.
\end{itemize}

\paragraph{Error Reporting}
%% Ivory's The use of advanced type-system features can produce elaborate and
%% confusing type-error messages to the user. Idris~\cite{christiansen2014reflect}
%% allows the programmer to supply error handlers, which are given a data structure
%% representing the compiler's error message and can rewrite it to insert
%% domain-specific knowledge. Unfortunately, GHC Haskell does not currently support
%% such a feature.

%% Thus, the EDSL author can provide custom type-error
%% messages using language the application programmers will understand.

Dynamically reporting errors in an EDSL is made difficult by the lack of
source locations. Consider the Ivory expression \hbox{\cd{x / y},} which induces a
runtime assertion \cd{y != 0}, inserted into the generated back-end (e.g., C code). If the assertion fails, we would like to include
a \emph{Haskell} source location in the error message to direct the programmer to the source of
the error. But how are we to obtain the source location in the first place?
Ivory programs are comprised of Haskell expressions and Haskell is
pure--functions cannot depend on their call-site--so we must look outside the
language proper to obtain source locations.

A common approach for writing location-aware functions in Haskell, epitomized by
the \cd{file-location} package~\cite{file-location}, is to use a Template
Haskell splice to query the compiler for the current source location and insert
it into the AST. This approach did not appeal to us as Template Haskell incurs
considerable syntactic overhead and is incompatible with the concrete syntax
described above, and we already had 10+ KLoC of Ivory implementation which would need to be
rewritten.

Instead, we opted to write a compiler plugin that rewrites GHC's intermediate
representation to add the source locations. To do so, we extended the \cd{Ivory} statement type with a new
\cd{Location} constructor that contains a location in the Haskell source,
essentially a special type of comment. The plugin then extracts the source
locations from GHC and wraps all actions in the \cd{Ivory} monad with a
\cd{withLocation} function that emits a \cd{Location} statement before executing
the wrapped action.

The approach limits the location granularity to
lines rather than columns, but it requires only modest changes to the
Ivory compiler and--importantly--no changes to Ivory code. As the plugin
itself requires little knowledge of Ivory, we have abstracted it out into a
separate package~\cite{ghc-srcspan-plugin} that can be reused by other projects.
Furthermore, we have since submitted a lightweight extension to GHC that allows
functions to request their call-site by taking a special implicit
parameter~\cite{lewis2000implicit} as an argument, which should become available
in the 7.12 release.
