\section{A Safe-C EDSL}
\label{sec:edsl}

Here, we remark briefly on some of the advantages and disadvantages of using an
EDSL in our domain. In a previous short paper, we explored some of the benefits
of using an EDSL for embedded system development~\cite{smaccm}. We will not
repeat those claims here, but rather describe new problems we have observed as
well as their mitigations.

A benefit of the EDSL approach is that it relieves the developer from having to
define and implement a front-end syntax. However, that also generally means that
only users of the host language will be attracted to using the
EDSL. We want C/C++ developers to use Ivory! Consequently, we implemented a
C-like concrete syntax for Ivory. We have particularly used the concrete syntax
in our work with Boeing, mentioned in Section~\ref{sec:introduction}.

Implementing a concrete syntax does not mean we have to abandon our EDSL. Using
Template Haskell~\cite{} and quasiquotation~\cite{}, 


\lee{Focus on some of the benefits/detriments of an EDSL for a safe-C
  language. Try not to overlap too much with last year's paper.}
\sjw{We should talk about what we borrow from the Haskell type checker}


\paragraph{Error Reporting}
A great strength of EDSLs is that you get to reuse the host language's parser,
type-checker, code-generator, etc. However, these benefits come at a cost to
error reporting, both static and dynamic. 

Statically, the use of advanced type-system features, e.g.\ higher-rank
polymorphism to track memory regions, can in practice produce elaborate and
confusing type-error messages. Idris~\cite{christiansen2014reflect} allows the
programmer to supply error handlers, which are given a data structure
representing the compiler's error message and can rewrite it to insert
domain-specific knowledge. Thus, the EDSL author can provide custom type-error
messages using language the application programmers will understand.
Unfortunately, GHC Haskell does not currently support such a feature.

Dynamically reporting errors in an EDSL is also made difficult by the lack of
source locations. Consider the Ivory expression \hbox{\cd{x / y},} which induces a
runtime assertion \cd{y != 0}. If the assertion fails, we would like to include
a source location in the error message to direct the programmer to the source of
the error. But how are we to obtain the source location in the first place?
Ivory programs are comprised of Haskell expressions and Haskell is
pure--functions cannot depend on their call-site--so we must look outside the
language proper to obtain source locations. A common approach for writing
location-aware functions in Haskell, epitomized by the \cd{file-location}
package~\cite{file-location}, is to use a Template Haskell splice to query the
compiler for the current source location and insert it into the AST. This
approach did not appeal to us as Template Haskell incurs considerable syntactic
overhead, and we already had 10+ KLoC of Ivory which would all need to be
rewritten.

Instead we opted to write a compiler plugin that rewrites GHC's intermediate
representation to add the source locations. The implementation was
straightforward, we first extended the \cd{Ivory} statement type with a new
\cd{Location} constructor that just contains a location in the Haskell source,
essentially a special type of comment. The plugin then extracts the source
locations from GHC and wraps all actions in the \cd{Ivory} monad with a
\cd{withLocation} function that emits a \cd{Location} statement before executing
the wrapped action. While this effectively limits the location granularity to
lines rather than columns, our approach required only modest changes to the
Ivory compiler and--importantly--no changes to Ivory code. As the plugin
itself requires little knowledge of Ivory, we have abstracted it out into a
separate package~\cite{ghc-srcspan-plugin} that can be reused by other projects.
Furthermore, we have since submitted a lightweight extension to GHC that allows
functions to request their call-site by taking a special implicit
parameter~\cite{lewis2000implicit} as an argument, which should become available
in the 7.12 release.

\lee{Lee: describe line number insertions (overcoming a bad thing), module
  system (hack), the use of TH to make a concrete syntax, macros (e.g., standard
  lib---compare to Rust), other uses of macros, small compiler.}

%%% Local Variables: 
%%% mode: latex
%%% TeX-master: "ivory"
%%% End: 
