\section{A Safe-C EDSL}
\label{sec:edsl}

Here, we remark briefly on some of the advantages and disadvantages---and
mitigations that we have explored to them---in using an EDSL in our domain. In a
previous short paper, we explored some of the benefits of using an EDSL for
embedded system development~\cite{smaccm}. We will not repeat those claims here,
but rather describe new problems we have observed as well as their mitigations.

A benefit of the EDSL approach is that it relieves the developer from having to
define and implement a front-end syntax. However, that also generally means that
only users of the host language will be attracted to using the
EDSL. We want C/C++ developers to use Ivory! Consequently, we implemented a
C-like concrete syntax for Ivory. We have particularly used the concrete syntax
in our work with Boeing, mentioned in Section~\ref{sec:introduction}.

Implementing a concrete syntax does not mean we have to abandon our EDSL. Using
Template Haskell~\cite{} and quasiquotation~\cite{}, 



\lee{Focus on some of the benefits/detriments of an EDSL for a safe-C
  language. Try not to overlap too much with last year's paper.}

\lee{Lee: describe line number insertions (overcoming a bad thing), module
  system (hack), the use of TH to make a concrete syntax, macros (e.g., standard
  lib---compare to Rust), other uses of macros, small compiler.}
