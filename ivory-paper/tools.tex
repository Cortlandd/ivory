\section{Ivory Testing and Verification}
\label{sec:tools}

Ivory contains built-in tools to support high-assurance software
development. These tools include a correctness condition generator, a
SMT-based symbolic simulator, a theorem-prover translator, and a
QuickCheck engine for randomized testing~\cite{qc}. We describe each of
them briefly below.

% \sjw{er, what?}
% None of these tools specifically depend on Ivory being an EDSL, and their
% implementations are straight-forward.

\subsection{Correctness Conditions}
Some correctness properties such as arithmetic underflow and overflow
cannot be embedded in the Haskell type system. For this class of
properties the translation into the Ivory AST adds instrumentation
containing appropriate assertions. For example, from the function

\begin{code}
add_ex :: Def ('[Sint32, Sint32] :-> Sint32)
add_ex = proc "add_ex" $ \x y -> body $
  ret (x + y)
\end{code}
\noindent
the following function is generated
\begin{code}
int32_t add_ex(int32_t var0, int32_t var1)
{
    bool i_ovf0 = add_ovf_i32(var0, var1);
    COMPILER_ASSERTS(i_ovf0);
    return (int32_t) (var0 + var1);
}
\end{code}
\noindent
Note the addition of an overflow check, performed by the function
\cd{add\_ovf\_i32}, before the expression is evaluated. The result of
this check is handled by the macro
\cd{COMPILER\_ASSERTS}; the response when a property is
violated is platform-dependent, and may include logging, do-nothing, running a
recovery procedure, and so forth.

Ivory inserts correctness condition checks for the following conditions, as requested by
the user:
\begin{itemize}
\item no arithmetic underflow and overflow,
\item no division-by-zero,
\item no bit-shifts are greater than or equal to the value's width,
  (bit-shifts on signed integers are prevented statically by the type system)
\item no floating-point operations result in \cd{inf} or \cd{NaN} values.
\end{itemize}

These correctness conditions are added as assertions by the Ivory
front-end.  While the implementation is straightforward, we note that
care must be taken to not over-constrain the emitted program: 
 Ivory contains
the short-cutting expressions of conjunctions, disjunctions, and conditionals,
analogs of C's \cd{\&\&}, \cd{||}, and \cd{\_ ? \_:\_}, respectively. For these
expressions, the generated assertions must contain as a precondition that
short-cutting has not occurred so as not to be overly pessimistic. For example,
for the expression
\noindent
\begin{code}
x != 0 ? 3/x : 0
\end{code}
\noindent
the (tautological) correctness condition
\begin{code}
(x == 0) || (x != 0)
\end{code}
\noindent
is generated.


In addition, Ivory's EDSL implementation encourages the use of macros
which can result in large expressions. In this case, repeated
sub-expressions will result in repeated assertions.  Standard common
subexpression elimination greatly reduces the number of
instrumented assertions.

\subsection{Discharging Correctness Conditions}

Generated correctness conditions as well as general user assertions can be
discharged via testing or using model-checking or theorem-proving. We describe
Ivory's tooling for these approaches below.

\paragraph{Symbolic Simulation}

Ivory contains a prototype symbolic simulator built over
CVC4~\cite{cvc4} for verifying programs.  Given a partially annotated
program, the simulator attempts to verify any inline assertions and
postconditions.  We have used the symbolic simulator to analyze
various Ivory programs, having found an off-by-one bug in a ring
buffer and verified the correctness of a safety state-machine in
SMACCMPilot.

The symbolic simulator abstracts various domains. Floating point types
are abstracted as reals. However, fixed-width values are modeled
precisely as are arrays.

During analysis, inter-procedural calls are inlined or abstracted as
directed by the user. If they are abstracted, then the callee's
precondition is added as a verification condition at the call
site. The callee's postcondition is added to the set of invariants
following the call. Ivory allows procedures imported from C to be
summarized with pre- and postconditions.

Ivory programs are amenable to formal analysis as they guarantee the
absence of memory-safety errors, and contain no pointer arithmetic or
concurrency.  In addition, Ivory programs typically use unbounded
iteration only at the top-level, thus all loops of interest have
statically known bounds and can be unrolled.

% The simulator ensures that the collection of preconditions are satisfiable.

% The simulator is simple, providing only bounded model-checking, no support for
% concurrency, and no counter-example guided refinement, matching the simplicity
% of the language.

\paragraph{Theorem Proving}

Eakman~et~al.~\cite{ivory-acl2} implemented a theorem-prover back-end
for Ivory that targets ACL2~\cite{acl}. The theorem-prover performs
inter-procedural analysis, abstracting procedure calls by their
contracts, as with the symbolic simulator. The ACL2 backend, being
interactive, has the potential to support more complex assertions than
the symbolic simulator, at the expense of user-directed
proofs. Eakman~et~al. provide examples of the use of the
theorem-proving back-end.

\paragraph{Property-Based Testing}

Finally, Ivory contains a QuickCheck~\cite{qc} like property-based
testing framework.  The framework tests procedures by randomly
generating values both for their formal arguments as well as for
global values. These values are constrained to satisfy any
pre-conditions.  

While Ivory has an interpreter, the testing framework uses the C
backend to compile the tests.  This ensures that the tested code is
that which will be run in the final compiled artifact, eliminating any
potential bugs introduced by the semantic gap between interpreter and
the compiled binary.


