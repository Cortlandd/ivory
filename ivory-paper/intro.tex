\section{Introduction}
\label{sec:introduction}

\lee{Lee will take a stab at this. The main point of this paper is that there
  has been a long history of ``safe C'' languages that involved building new
  compilers. They allow low-level programming while ensuring type-safety. We
  show that with a modest set of generic type-level features, such a language
  can be built as an EDSL in which the safe-C type system is embedded in
  Haskell's type system, with a small number of exceptions. We'll also talk
  about some of the benefits and detriments of a safe C EDSL as opposed to a
  stand-alone compiler. In some sense, this is a follow-up to Stephanie's ICFP
  keynote showing what you can do with Haskell's dependent types (meaning
  decidable type inference in our usage).}

In 2014, we published an experience report describing our approach to building
high-assurance embedded systems, particularly focused on the use of
EDSLs~\cite{smaccm}. In that paper, we briefly introduced a language called
\emph{Ivory} for programming high-assurance embedded systems. While we briefly
discussed our user experience of Ivory there, our main application with the
language has been to write \emph{SMACCMPilot}, a full-featured autopilot for a
small unpiloted air vehicle that was the subject of a ``penetration test'' by
the U.S. Government. Here we describe the design and implementation of the
language.

Ivory follows in the footsteps of other ``safe C'' programming languages, such
as Cyclone~\cite{}, Rust~\cite{}, \lee{others?}. By safe C languages, we mean
languages that avoid many of the pitfalls of C, particularly related to memory
safety and undefined behavior, that are suitable for writing low-level code
(e.g., device drivers), and that provide low-level control over timing behavior
and memory usage. Ivory, like other safe C languages, has a minimal runtime
system.

\paragraph{Contributions}
In 2014, Stephanie Weirich gave a keynote address at the International
Conference on Functional Programming describing dependent programming using the
type-extensions of GHC's implementation of Haskell (which we will colloquially
refer to as ``GHC'' in the remainder of this
paper)~\cite{weirich-keynote}. Weirich describes how recent extensions to GHC
provide much of the power of dependently-typed programming, such as found in
Agda~\cite{}, Idris~\cite{}, or Coq~\cite{}. However, in GHC, a surprisingly
powerful subset of dependent typing features can be used while still enjoying
type-inference and decidable type-checking~\cite{dephaskell}.

Ivory exemplifies the use of GHC dependent types in a large, fully-featured
EDSL. A contribution of the language is to show that type checking for safe
programming involving the use of pointers, arrays, loops and local memory
allocation, can be embedded into GHC's type system.

After providing a brief introduction to the Ivory language in
Section~\ref{sec:ivory-overview}, we describe Ivory's embedding in GHC's type
system in Section~\ref{sec:ivory-embedding}, obviating the need for a writing a
custom type-checker or compiler for a safe C language.

Embedding a type system for a safe C language into GHC's type system is tricky
business. To gain confidence that our embedding is correct, we have formalized a
model of Ivory in the Isabelle theorem prover~\cite{}, and used the model to
formally prove progress and preservation properties for Ivory. In the process,
we have discovered extensions to Ivory that we initially were concerned may not
be type-safe but in fact are safe. We describe the formalization, proofs, and
extensions in Section~\ref{sec:semantics}.

Ivory goes beyond ensuring memory safety, the focus on most other safe C
programming languages, and also provides automated support for preventing errors
that result from other undefined behaviors in C (e.g., division by zero, left
bit-shifts by a negative value, etc.) as well as support for checking
user-provided assertions. Toward this end, Ivory supports writing user-supplied
assertions and pre- and post-conditions on functions, and includes a built-in
symbolic simulator targeting an SMT solver (CVC4~\cite{}), as well as an
theorem-prover back-end targeting ACL2~\cite{}. For automated testing, a
QuickCheck-like property-based test-case generator is integrated into
Ivory. These tools are described in Section~\ref{sec:tools}.

\lee{talk about location of Ivory, open source, size, etc.}
