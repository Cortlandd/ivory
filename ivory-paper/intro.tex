\section{Introduction}
\label{sec:introduction}

Recent reports of car-hacking via software flaws~\cite{}, unsafe
safety-critical software~\cite{}, and bugs in internet-of-things
devices~\cite{}, all point toward the need for safe low-level programming
languages. Languages like C or C++ are still the gold standard in embedded
system development given the low-level control they provide in terms of memory
usage and timing behavior. Unfortunately, these languages provide
little support for creating high assurance software --- they are
unsafe and unanalyzable.

In this paper we present the language \emph{Ivory}\footnote{%
Ivory is open-source (BSD3 license) and available at \url{ivory-lang.org}.}.
Ivory follows in the
footsteps of other ``safe C'' programming languages, like
Cyclone~\cite{cyclone}, BitC~\cite{bitc}, and Rust~\cite{rust}. By safe C
languages, we mean languages that avoid many of the pitfalls of C, particularly
related to memory safety and undefined behavior, while being suitable for
writing low-level code (e.g., device drivers). Ivory, like other safe C
languages, has a minimal runtime system.

Ivory is particularly designed for safety-critical embedded programming. Such a language
should guarantee memory safety, prevent most undefined behaviors, and provide
integrated testing and verification tools. Additional conventions for safe
embedded systems such as those in use at NASA's Jet Propulsion
Laboratory~\cite{mars}, are enforced by Ivory's type system. The major
restrictions enforced by Ivory include no dynamic memory allocation, all loops
having constant upper bounds, avoiding machine-dependent types (e.g., \cd{int}),
do not use nested pointers.

Ivory's implementation, however, is unique compared to previous safe C
languages: Ivory is implemented as an embedded domain-specific
language (EDSL) within Haskell.  In addition to the
benefits of rapid language development, this gives Ivory a powerful
templating system---the language Haskell---allowing low-level
programs to be written in a high-level style, despite some of the language restrictions.

Ivory's type system is shallowly embedded within Haskell's type
system, taking advantage of the extensions provided
by GHC~\cite{dephaskell}.  Thus, well-typed Ivory programs are
guaranteed to produce memory safe executables, all without writing a
stand-alone type-checker.

In contrast, the Ivory syntax is \emph{deeply} embedded within
Haskell.  This novel combination of shallowly-embedded types and
deeply-embedded syntax permits ease of development without sacrificing
the ability to develop various back-ends and verification tools: in
addition to the generation of embedded C for compilation, the Ivory
language suite includes an integrated SMT-based symbolic simulator and
a theorem-prover back-end.  All these back-ends share the same AST:
Ivory verifies what it compiles.


Ivory is not a toy language: we have used Ivory to write
\emph{SMACCMPilot}~\cite{smaccm}, a full-featured high-assurance
autopilot for a small unpiloted air vehicle.  Furthermore, Boeing has
used Ivory to implement a level-of-interoperability for
Stanag~4586~\cite{stanag}, a unpiloted air vehicle communications
standard. We know of a few additional small projects by other
developers in Ivory, as well.  There are well over 100~KLoC of Ivory in existence.

\sjw{Ivory uses regions for memory management.}

\sjw{move to contributions bit?}

\paragraph{Contributions}
In 2014, Stephanie Weirich gave an ICFP keynote describing dependently-typed programming
using recent type and language extensions of GHC's implementation of Haskell
(which we will colloquially refer to as ``GHC'' in the remainder of this
paper)~\cite{weirich-keynote}. Weirich describes how recent extensions to GHC
provide much of the power of dependently-typed programming, such as found in
Agda~\cite{agda}, Idris~\cite{idris}, or Coq~\cite{coq}. However, in GHC, a
surprisingly powerful subset of dependent typing features can be used while
still enjoying type-inference and decidable type-checking~\cite{dephaskell}.

Ivory exemplifies the use of GHC dependent types in a large, fully-featured
EDSL. The language demonstrates that type checking for safe systems programming
can be embedded into GHC's type system, verifying properties involving
pointers, arrays, loops and local memory allocation. Indeed, Ivory's type
system goes beyond mere memory safety and tracks control-flow effects at host-language
compile-time.

After providing a brief introduction to the Ivory language in
Section~\ref{sec:ivory-overview}, we describe Ivory's embedding in GHC's type
system in Section~\ref{sec:ivory-embedding}. We highlight the aspects of the
language particularly relevant to memory-safety (e.g., pointers, structures, and
memory allocation). We also highlight shortcomings of the approach, describing
aspects of the language that cannot be checked by the host language's type
system (e.g., Ivory's module system).

Embedding a type system for a safe C language into GHC's type system is tricky
business. To gain confidence that our embedding is correct, we formalized a
model of Ivory in the Isabelle theorem prover~\cite{isabelle}, and used the model to
formally prove progress and preservation properties for Ivory. In the process,
we discovered minor bugs in Ivory's type embedding in GHC as well as
generalizations to Ivory that still preserve safety. We describe the
formalization, proofs, and extensions in Section~\ref{sec:semantics}.

Ivory goes beyond ensuring memory safety, the focus of most other safe C
programming languages, and also provides automated support for preventing errors
that result from other undefined behaviors in C (e.g., division by zero, left
bit-shifts by a negative value, etc.) as well as support for checking
user-provided assertions. Toward this end, Ivory supports writing user-supplied
assertions and pre- and post-conditions on functions, and includes a built-in
symbolic simulator targeting an SMT solver (CVC4~\cite{cvc4}), as well as an
theorem-prover back-end targeting ACL2~\cite{acl2}. For automated testing, a
QuickCheck-like property-based test-case generator is integrated into
Ivory. These tools are described in Section~\ref{sec:tools}. In
Section~\ref{sec:edsl}, we discuss some of the issues and our mitigations with
using a large EDSL for embedded programming projects.

We describe related work in safe C language and EDSLs in
Section~\ref{sec:related-work} and provide concluding remarks in
Section~\ref{sec:conclusion}.



