\section{Introduction}
\label{sec:introduction}

\lee{also mention in contributions that we've worked on edsl usability?}

Recent reports of car-hacking via software flaws~\cite{}, unsafe software
safety-critical software~\cite{}, and bugs in internet-of-things
devices~\cite{}, all point toward the need for safe low-level programming
languages. Languages like C or C++ are still the golden standard in embedded
system development given the low-level control they provide in terms of memory
usage and timing behavior. But C/C++ is wrought with undefined behavior with
poor support for formal analysis.

In this paper we present the language \emph{Ivory}.  Ivory follows in the
footsteps of other ``safe C'' programming languages, like
Cyclone~\cite{cyclone}, BitC~\cite{bitc}, and Rust~\cite{rust}. By safe C
languages, we mean languages that avoid many of the pitfalls of C, particularly
related to memory safety and undefined behavior, while being suitable for
writing low-level code (e.g., device drivers). Ivory, like other safe C
languages, has a minimal runtime system.

Ivory's implementation, however, is unique compared to previous safe C
languages. Ivory is implemented as an embedded domain-specific langauge
(EDSL)~\cite{edsl} within Haskell. Ivory's type system is (shallowly) embedded
within Haskell's type system, including dependently-typed extensions provided by
GHC~~\cite{dephaskell}. Without writing a stand-alone type-checker, Ivory types
guarantee memory safety of Ivory programs at \emph{Haskell} compile time.

However, the Ivory language itself is deeply embedded within Haskell, easing the
ability to develop various back-ends and verification tools. Not only does Ivory
generate embedded C, it contains an integrated SMT-based symbolic simulator, and
a theorem-prover back-end. The combination of a shallowly-embedded types and
deeply embedded values is novel and pragmatic.

Ivory is not a toy language: we have used Ivory to write
\emph{SMACCMPilot}~\cite{smaccm}, a full-featured high-assurance
autopilot for a small unpiloted air vehicle.  Furthermore, Boeing has
used Ivory to implement a level-of-interoperability for
Stanag~4586~\cite{stanag}, a unpiloted air vehicle communications
standard. We know of a few additional small projects by other
developers in Ivory, as well.  There are well over 100 KLoC of Ivory in existence.

%% maybe remove?  I wanted to talk briefly about the EDSL aspect, etc.
%% Question this answers: why should I care?  Might be answered in the
%% next bit ...
\sjw{more of a summary paragraph, probably rephrase or remove?}

\paragraph{Contributions}
In 2014, Stephanie Weirich gave a keynote address at the International
Conference on Functional Programming describing dependent programming using
recent type and language extensions of GHC's implementation of Haskell (which we
will colloquially refer to as ``GHC'' in the remainder of this
paper)~\cite{weirich-keynote}. Weirich describes how recent extensions to GHC
provide much of the power of dependently-typed programming, such as found in
Agda~\cite{agda}, Idris~\cite{idris}, or Coq~\cite{coq}. However, in GHC, a surprisingly
powerful subset of dependent typing features can be used while still enjoying
type-inference and decidable type-checking~\cite{dephaskell}.

Ivory exemplifies the use of GHC dependent types in a large, fully-featured
EDSL. The language demonstrates that type checking for safe systems programming
can be embedded into GHC's type system, verifying properties involving
pointers, arrays, loops and local memory allocation. Indeed, Ivory's type
system goes beyond mere safety and tracks effects, such as control flow
operators like \cd{break} and \cd{return} statements.

After providing a brief introduction to the Ivory language in
Section~\ref{sec:ivory-overview}, we describe Ivory's embedding in GHC's type
system in Section~\ref{sec:ivory-embedding}, obviating the need for a writing a
custom type-checker or compiler for a safe C language.

Embedding a type system for a safe C language into GHC's type system is tricky
business. To gain confidence that our embedding is correct, we have formalized a
model of Ivory in the Isabelle theorem prover~\cite{isabelle}, and used the model to
formally prove progress and preservation properties for Ivory. In the process,
we have discovered minor bugs in Ivory's type embedding in GHC as well as
generalizations to Ivory that still preserve safety. We describe the
formalization, proofs, and extensions in Section~\ref{sec:semantics}.

Ivory goes beyond ensuring memory safety, the focus on most other safe C
programming languages, and also provides automated support for preventing errors
that result from other undefined behaviors in C (e.g., division by zero, left
bit-shifts by a negative value, etc.) as well as support for checking
user-provided assertions. Toward this end, Ivory supports writing user-supplied
assertions and pre- and post-conditions on functions, and includes a built-in
symbolic simulator targeting an SMT solver (CVC4~\cite{cvc4}), as well as an
theorem-prover back-end targeting ACL2~\cite{acl2}. For automated testing, a
QuickCheck-like property-based test-case generator is integrated into
Ivory. These tools are described in Section~\ref{sec:tools}.

Ivory is open-source (BSD3 license) and available at \url{ivory-lang.org}. 
