\section{Ivory Embedding}
\label{sec:ivory-embedding}

\lee{This section will be the heart of the paper, focusing on how we have
  embedded Ivory in Haskell's type system. I expect this to be 3-5 pages and
  probably the most technical. We want to be detailed, but describe things
  general enough that ML, Agda, etc. programmers can understand.
}

\subsection{The Ivory Monad}

\lee{Lee: describe the Ivory monad (not too much to say), then describe
  effects. Make sure to make clear these are monad effects like are popular in
  the literature. Mention inspiration from
  \url{http://www.doc.ic.ac.uk/~wlj05/files/Deconstraining.pdf}, but didn't
  follow implementation.}

\lee{Also talk about the use of typeclasses \cd{IvoryVar}, \cd{IvoryExpr},
  etc. and how this allows the user to define custom data, safely.}

\subsection{References and Allocation}
Ivory manages allocated data through the use of non-nullable references.
References are represented using the \cd{Ref} type, which takes two parameters:
the scope that it was allocated in, and the type of the memory area it points
to.

All references in Ivory are allocated in one of two scopes, \cd{Global}, or a
fresh local scope, unique to the function that the allocation takes place
within.  References with \cd{Global} scope are allocated through the use of the
\cd{area} top-level declaration, and references allocated within a function are
allocated through the use of the \cd{local} function.  Both can take
initializers, defaulting to zero-initialization when they are omitted.

While references allocated in the \cd{Global} scope are allocated when the
program starts, references allocated in the local scope of a function are stack
allocated, and freed as soon as the function returns.

\subsection{Memory Areas}
In Ivory, data pointed to by references and pointers is described by the {\tt
Area} kind, following \cite{memareas}.  This kind contains four types: {\tt
Stored}, \cd{Struct}, \cd{Array}, and \cd{CArray}.

\begin{figure}[h]
\begin{code}
data Area k = Array Nat (Area k)
            | CArray (Area k)
            | Struct Symbol
            | Stored k
\end{code}
\caption{The definition of the \cd{Area} kind}
\end{figure}

\paragraph{Stored values}
The simplest type of memory area is a single base type, lifted to the \cd{Area}
kind by the use of the \cd{Stored} type constructor.  For example, the type of
a single stored \cd{Sint32} would simply be \cd{Stored Sint32}.

\paragraph{Structs}
The struct type identifies an area of memory as being structured by the
corresponding struct declaration from a use of the ivory quasiquoter.  For
example, if a region of memory is typed using the struct definition from figure
\ref{example-struct}, it would have type \cd{Struct "a"}.

\begin{figure}[h]
\begin{code}
[ivory| struct a { field1 :: Stored Sint32
                 , field2 :: Struct "b"
                 }
      |]
\end{code}
\caption{An example struct definition}
\label{example-struct}
\end{figure}

Also introduced by the struct declaration are field labels.  Field labels allow
for indexing into a memory area, producing a reference to the value contained
within the struct.  For example, using the struct definition from figure
\ref{example-struct}, there would be two labels introduced: \cd{field1}, and
\cd{field2}, and will have types \cd{Label "a" (Stored Sint32)} and \cd{Label
"a" (Struct "b")}, respectively.

Using a struct label to select the field of a structure requires the use of the
\cd{(\mytilde>)} operator, which expects a reference to a structure as its first
argument, and a compatible label as its second.  In figure
\ref{example-struct-label}, the \cd{(\mytilde>)} operator is used with a reference to
an ``a'' struct, with the \cd{field1} label, producing a new reference of type
\cd{Ref Global (Stored Sint32)}

\begin{figure}[h]
\begin{code}
example :: Ref Global (Struct "a")
        -> Ref Global (Stored Sint32)
example ref = ref \mytilde> field1
\end{code}
\caption{Struct field indexing}
\label{example-struct-label}
\end{figure}

\paragraph{Arrays}
Arrays in Ivory take two type parameters: the length of the array as a
type-level natural number, and the area type of its elements.  For example, an
array of 10 \cd{Sint32} would have the type \cd{Array 10 (Stored
Sint32)}.  Indexing into arrays is accomplished through the use of the \cd{(!)}
operator, and a valid index.

An index into an array has the type \cd{Ix}, which is parameterized by the size
of the array that it is indexing into.  The \cd{Ix n} type will only hold
values between zero and \cd{n}, which allows us to avoid run-time array bounds
checks, as in \ref{memareas}.

Additionally, we define a combinator \cd{arrayMap}, 

\lee{Trevor: Introduce the use of type-level nats here, and how they are used to
  enforce read/write safety for arrays. I'd discuss \cd{arrayMap} and \cd{Ix}
  here, too. (Maybe discuss possible extensions with decidable type-level
  arithmetic). Maybe mention \cd{CArray}s, too.}

\subsection{Procedures}
\lee{Trevor: describe the use of type-level lists to define procedures. Describe
the use of a type-class to ensure procedure calls in Ivory are type-correct.}

\subsection{Bit-Data}
\lee{James: talk about using TH to define bitdata.}

\subsection{Module System}
\lee{Lee: describe the module system, warts and all.}
